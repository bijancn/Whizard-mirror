\documentclass[12pt,a4paper]{article}
\usepackage{amsmath}
\usepackage{emp}
\setlength{\unitlength}{1mm}
\begin{document}
\begin{empfile}
\begin{figure}
  \begin{center}
    \begin{emp}(80,60)
      pair c; c = (.5w,.5h);
      path k[], p[], a[], b[];
      p1 = (0,.5h) -- c; p2 = c -- (.4w,h); p0 = c -- (w,.5h);
      k1 = (w,.7h) -- c; k2 = c -- (w,.2h);
      pickup pencircle scaled .5pt;
      draw (0,.5h)--(w,.5h);
      % a1 = (point .7 of p0){up} .. {direction .5 of k1 rotated 270}(point .3 of k1);
      % a2 = (point .7 of p0){down} .. {direction .5 of k2 rotated 270}(point .6 of k2);
      b1 = fullcircle rotated 180 scaled .6w shifted c;
      b2 = fullcircle rotated 180 scaled .7w shifted c;
      a1 = subpath (xpart (b1 intersectiontimes k1), xpart (b1 intersectiontimes p0)) of b1;
      a2 = subpath (xpart (b2 intersectiontimes k2), xpart (b2 intersectiontimes p0)) of b2;
      draw a1; label.rt (btex $\alpha_0$ etex, point (.5length a1) of a1);
      draw a2; label.lft (btex $\theta$ etex, point (.5length a2) of a2);
      pickup pencircle scaled 1.5pt;
      drawarrow subpath (0,.9) of p1; label.top (btex $p$ etex, point .5 of p1);
      drawarrow subpath (0,.9) of k1; label.top (btex $k$ etex, point .5 of k1);
      drawarrow subpath (.1,1) of p2; label.lft (btex $p'$ etex, point .5 of p2);
      drawarrow subpath (.1,1) of k2; label.bot (btex $k'$ etex, point .5 of k2);
    \end{emp}
  \end{center}
  \caption{\label{sec:kinematics}%
    Kinematics.}
\end{figure}
\begin{subequations}
\begin{align}
  k &= (\omega_0; -\omega_0\sin\alpha_0, 0, -\omega_0\cos\alpha_0) \\
  k' &= (\omega; -\omega\sin\theta, 0, \omega\cos\theta) \\
  p &= (E; 0, 0, \sqrt{E^2-m_e^2})
\end{align}
\end{subequations}
The kinematics is a little bit hairier than usual,
because~$\alpha_0\not=0$.  One approach is to boost to the center of
mass system and back, but it's far easier to use the invariants:
\begin{subequations}
\begin{align}
  s &= (k+p)^2 = m_e^2 + 2\omega_0(E+\sqrt{E^2-m^2}\cos\alpha_0) \\
  t &= (k'-k)^2 = - 2\omega_0\omega(1+\cos(\alpha_0+\theta)) \\
  u &= (k'-p)^2 = m_e^2 - 2\omega(E-\sqrt{E^2-m^2}\cos\theta)
\end{align}
\end{subequations}
and the Mandelstam relation
\begin{equation}
  s + t + u = 2m_e^2
\end{equation}
Then
\begin{equation}
 \omega =
   \frac{\omega_0(E+\sqrt{E^2-m_e^2}\cos\alpha_0)}
        {E-\sqrt{E^2-m_e^2}\cos\theta+\omega_0(1+\cos(\alpha_0+\theta))}
\end{equation}
Expand
\begin{subequations}
\begin{align}
  \cos\theta &= 1 - \frac{1}{2}\theta^2 + O(\theta^4) \\
  \cos(\alpha_0+\theta) &=
    \left(1 - \frac{1}{2}\theta^2\right)\cos\alpha_0 - \theta\sin\alpha_0
       + O(\theta^3) \\
  \sqrt{E^2-m_e^2} &= E - \frac{m_e^2}{2E} + O(m_e^4)
\end{align}
\end{subequations}
Also
\begin{equation}
  \cos\alpha_0 = 2\cos^2\left(\frac{\alpha_0}{2}\right) - 1
\end{equation}
and we define
\begin{equation}
  \cos^2\left(\frac{\alpha_0}{2}\right) = \frac{xm_e^2}{4E\omega_0}
\end{equation}
Then
\begin{equation}
 \omega = \frac{xE}{1 + x + \frac{E^2}{m^2}\theta^2}
\end{equation}
or
\begin{equation}
 \omega = \frac{xE}{1 + x} \frac{1}{1 + \left(\frac{\theta}{\theta_0}\right)^2}
\end{equation}
with
\begin{equation}
  \theta_0 = \frac{m^2}{E^2}(1+x)
\end{equation}

\end{empfile}
\end{document}



