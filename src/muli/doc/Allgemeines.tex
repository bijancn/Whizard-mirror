\part{Allgemeines}
\begin{figure}
  \includegraphics[scale=0.75,angle=90]{uml-module-tree-1.mps}
\end{figure}
\chapter{Nomenklatur}
\begin{figure}
  \includegraphics{diagrams-1.mps}
  \includegraphics{diagrams-2.mps}
  \caption{\label{fig:nomen:had}Links: Impulsvariablen $P_x$ der Remnants, Impulsvariablen $\hat{p}_x$ der Partonen und Flavorindizes $a^{(k)},b^{(k)},c^{(k)},d^{(k)}$ der Partonen in der $k$-ten Iteration des Multiple Interactions Algorithmus. Rechts: Die Prozeduren zur Generierung der Ereignisse kennen üblicherweise nicht die Ordnungszahl $k$ nur einen Teil der Impulsinformation, nämlich die hadronischen Impulsanteile $X$ mit $P^{(k)}=XP$ und die partonischen Impulsanteile $\xi$ mit $\hat{p}^{(k)}=\xi P^{(k)}=x P$. Anstatt der Flavorindizes $a^{(k)},b^{(k)},c^{(k)},d^{(k)}$ ist die festgelegte Position in dem Flavorquadrupel eingetragen.}
\end{figure}
\section{n-te Wechselwirkung}
Durch den Algorithmus werden iterativ harte, partonische, treelevel, QCD $2\rightarrow 2$ Wechselwirkungen mit absteigenden Wechselwirkungsskalen $\pperp^{(n)}$ generiert. Variablen, die nach jeder harten Wechselwirkung einen neuen Wert erhalten, führen die Ordnungszahl $n$ der aktuellen Wechselwirkung hochgestellt in runden Klammern, um Verwechslungen mit Potenzen zu vermeiden.

Diese Ordnungszahl wird mit $k$ bezeichnet, wenn sie sich nicht auf die aktuelle Wechselwirkung bezieht, sondern Summations- oder Produktindex über alle bisherigen Wechselwirkungen ist. Die Ordnungszahl der letzten Wechselwirkung wird mit $N$ notiert. Bevor eine harte Wechselwirkung stattfindet, ist die Ordnungszahl gleich Null und kann weggelassen werden.
\section{Impulse}
Die Viererimpulse der Remnants sind $P_1^{(n)}$ bzw. $P_2^{(n)}$ für das erste bzw. das zweite Proton. Die Viererimpulse der Partonen sind $\hat{p}_1^{(n)}$ bzw. $\hat{p}_2^{(n)}$ für das erste bzw. das zweite Proton. Die Viererimpulse der Teilchen im Endzustand der partonischen Wechselwirkung sind für MulI nicht von Bedeutung.

Die Kinematik eines partonischen Ereignisses wird vollständig durch das kartesische Quadrupel $p_{\cart}^{(n)}=\left[x_1^{(n)},x_2^{(n)},\pperp^{(n)};s^{(n)}\right]$ bzw. das hyperbolische Quadrupel $p_{\hyp}^{(n)}=\left[h_1^{(n)},h_2^{(n)},h_3^{(n)};s^{(n)}\right]$ definiert. Die entsprechende Koordinatentransformation ist in \eqref{eq:all:imp:trafo} angegeben. Es wird nicht zwischen der Bjorken-Scaling-Variable $x$ und dem Impulsanteil mit $xP=\hat{p}$ unterschieden. Wir nehmen an, dass die kinetische Energie der Protonen viel größer als die Ruhemasse der Protonen ist. In diesem Grenzwert stimmen beide Variablen überein.
\section{Flavor}
Quarks und Gluonen der $n$-ten harten Wechselwirkung haben Flavorindizes $a^{(n)}, b^{(n)},c^{(n)},d^{(n)}$, wobei $a$ das Flavor des Partons aus dem Remnant 1 und $b$ das Flavor des Partons aus dem Remnant 2 ist. Wenn nicht anders angegeben, wird das LHAPDF-Schema verwendet, mit $[\overline{t},\overline{b},\overline{c},\overline{s},\overline{u},\overline{d},g,d,u,s,c,b,t]=[-6,-5,-4,-3,-2,-1,0,1,2,3,4,5,6]$. 
\section{Strukturfunktionen}
\mip{Strukturfunktionen $f(x,\mu)$ sind in diesem Dokument synonym zu Flavor-Strukturfunktionen, die Impuls-Strukturfunktionen ergeben sich dann aus $xf(x,\mu)$. Im Gegensatz dazu liefert evolvePDF aus LHAPDF die Impulsstrukturfunktion. MulI hat diesbezüglich die gleiche Konvention wie die Builtin-PDFs aus WHIZARD, welche ebenfalls Flavor-PDFs liefern.}

Da beide Remnants eine verscheidene Historie haben können, sind im Allgemeinen auch beide Remnant-Strukturfunktionen verschieden. Die Zugehörigkeit wird durch den Flavorindex $a$ für das erste und $b$ für das zweite Proton notiert. Wir verzichten auf die Indizierung der Flavorindizes, also $a^{(n)}\rightarrow a$, da die Strukturfunktionen bereits Ordnungsindizes ${(n)}$ haben. Wir erhalten $f_a^{(n)}\big(x_1^{(n)},\mu_F^{(n)}\big)$ für das erste Proton und $f_b^{(n)}\big(x_2^{(n)},\mu_F^{(n)}\big)$ für das zweite Proton.

\wip{Derzeit sind im Modul \ModuleRef{muli\_remnant} Proton-Strukturfunktionen fest implementiert, es können also ohne Eingriff in den Code z.B. keine Proton-Antiproton Streuungen generiert werden. Allerdings sollte die Verallgemeinerung auf Hadronen mit maximal zwei verschiedenen Valenzquarks kaum Probleme bereiten, da die Infrastruktur des Moduls nicht geändert werden muss.}

\section{Wirkungsquerschnitte}
Wie die Strukturfunktionen ändern sich auch die Wirkungsquerschnitte $\sigma$ mit jeder Iteration. Allerdings können wir die Abhängigkeit komplett in die Änderung der invarianten Masse $s^{(n)}$ der $n$-ten Iteration absorbieren. Wir notieren also keinen Ordnungsindex $(n)$, sondern fügen die invariante Masse als Parameter durch ein Semikolon getrennt in die Liste der Argumente ein. Wir erhalten für den hadronischen Wirkungsquerschnitt ${\sigma}$
\begin{equation}
  {\sigma}_{ab\rightarrow cd}^{(n)}\left(x_1^{(n)},x_2^{(n)},\pperp^{(n)}\right)
  ={\sigma}_{ab\rightarrow cd}\left(x_1^{(n)},x_2^{(n)},\pperp^{(n)};s^{(n)}\right).
\end{equation}
Der hadronische Wirkungsquerschnitt $\sigma$ bezieht sich auf das Streuereignis, wie in Abbildung \ref{fig:nomen:had} dargestellt. Der partonische Wirkungsquerschnitt hingegen ist $\widehat{\sigma}$.
\section{Übersicht}
\begin{align}
  s&=s^{(1)}=P_1\cdot P_2\\
  s^{(n)}&=P_1^{(n)}\cdot P_2^{(n)}\\
  \hat{p}_1^{(n)}&=P_1^{(n)}x_1^{(n)}\\
  P_1^{(n+1)}&=P_1^{(n)}\big(1-x_1^{(n)}\big)\\
  X^{(n)}&=\prod_{k=1}^{n}\left(1-x^{(k)}\right)\\
  P_1^{(n+1)}&=X_1^{(n)}P^{(1)}\\
  \pperp&=\frac{\hat{t}\hat{u}}{\hat{s}}
\end{align}
\chapter{Der Algorithmus}
Der Algorithmus ist in meiner Dissertation in Kapitel 5 bereits dokumentiert, deswegen werde ich hier nicht alle Aspekte wiederholen. In der Dissertation wird allerdings nicht sorgfältig zwischen fertigen und geplanten Eigenschaften getrennt, deswegen gebe ich hier einen groben Überblick über den aktuellen Stand.

MulI wird derzeit ausschließlich von dem shower\_interface aus dem interleaved Branch aus dem schmidtboschmann Verzeichnis des WHIZARD-Repositories aufgerufen. Es ist noch kein MulI-Code in den WHIZARD-Core übertragen worden, stattdessen sind alle relevanten Daten in einem erweiterten Datentyp \TypeRef{muli\_type} gekapselt. \TypeRef{muli\_type} stellt ebenfalls eine vollständige Schnittstelle bereit, um MPI zu generieren und Remnant-PDFs abzurufen. Die derzeit verwendeten Methoden dieser Schnittstelle sind in Tabelle \ref{tab:all:interface} aufgeführt.
\begin{table}
\begin{center}
\begin{tabular}{ll}
Generischer Name & Spezifischer Name\\
\midrule
\TbpRef{muli\_type}{initialize} & \ProcRef{muli\_initialize}\\
\TbpRef{muli\_type}{restart} & \ProcRef{muli\_restart}\\
\TbpRef{muli\_type}{finalize} & \ProcRef{muli\_finalize}\\
\TbpRef{muli\_type}{apply\_initial\_interaction} & \ProcRef{muli\_apply\_initial\_interaction}\\
\TbpRef{muli\_type}{generate\_gev2\_pt2} & \ProcRef{muli\_generate\_gev2\_pt2}\\
\TbpRef{muli\_type}{generate\_partons} & \ProcRef{muli\_generate\_partons}\\
\TbpRef{qcd\_2\_2\_type}{get\_color\_correlations} & \ProcRef{qcd\_2\_2\_get\_color\_correlations}\\
\TbpRef{muli\_type}{replace\_parton} & \ProcRef{muli\_replace\_parton}\\
\TbpRef{muli\_type}{get\_parton\_pdf} & \ProcRef{muli\_get\_parton\_pdf}\\
\TbpRef{muli\_type}{get\_momentum\_pdf} & \ProcRef{muli\_get\_momentum\_pdf}
\end{tabular}
\caption{\label{tab:all:interface}Die Methoden der MulI Schnittstelle für den Interleaved-Algorithmus.}
\end{center}
\end{table}
\begin{figure}
\begin{center}
\includegraphics{uml-1.mps}
\end{center}
\caption{Flussdiagramm des Interleaved-Algorithmus. Eingetragen sind ausschließlich die Aufrufe der MulI-Schnittstelle, mit Ausnahme von get\_parton\_pdf und get\_momentum\_pdf, da diese nur für Interna des Partonshowers relevant sind und das Diagramm unnötig kompliziert machen würden. Die Generierung der Showerskalen und -Teilchen ist hier nicht dargestellt. Parallel zu generate\_gev2\_pt2 wird von dem ISR-Modul eine Showerskala $t$ generiert und unmittelbar vor replace\_parton wird von dem ISR-Modul ein neues Showerteilchen generiert, dass durch replace\_parton in die Beschreibung des Remnants aufgenommen wird. $m_E$ ist die Zahl der zu generierenden Events.}
\end{figure}
\section{Stratified Sampling}
\label{sec:all:alg:stra}
Die Wahrscheinlichkeit dafür, dass die Wechselwirkung aus dem Stratum $\{\alpha,\beta\}$ mit der größten Skala $\pperp\leq\pperp^{(n-1)}$ bei der Skala $\pperp^{(n-1)}$ stattfindet, ist durch 
\begin{equation}
\mathcal{P}_{\text{next},a,b}^{(n)}\left(\pperp^{(n)};\pperp^{(n-1)}\!\!,s^{(n)}\right):=\exp\left[W_a^{(n)}W_b^{(n)}
\left[
\mathcal{S}_{\alpha\beta}\left(\pperp^{(n)};s^{(n)}\right)
-
\mathcal{S}_{\alpha\beta}\left(\pperp^{(n-1)};s^{(n)}\right)
\right]
\right]\label{eq:imp:pnextab}
\end{equation}
gegeben. $s^{(n)}$ ist die invariante Masse des Remnant-Remnant-Systems, $W_\alpha^{(n)}$ und $W_\beta^{(n)}$ sind die Wichtungsfaktoren des Stratums $\alpha$ bzw. $\beta$ und $\mathcal{S}_{\alpha\beta}$ ist das Stammstratum mit
\begin{equation}
  \mathcal{S}_{\alpha\beta}\left(\pperp^{(n)};s^{(n)}\right):=\int_{\pperp^{\max}}^{\pperp^{(n)}}\der\pperp\overline{S}_{\alpha\beta}\left(\pperp;s^{(n)}\right)\label{eq:all:strati_root_def}.
\end{equation}
Das Stammstratum ist demnach eine negative Stammfunktion des integrierten Stratums $\overline{S}_{\alpha\beta}$ mit ${\pperp^{\max}}=s/4$ und
\begin{equation}
  \overline{S}_{\alpha\beta}\left(\pperp;s^{(n)}\right):=\int_{x_{\min}}^{1}\der x_1\int_{x_{\min}}^{1}\der x_2\ S_{\alpha\beta}\left(x_1,x_2,\pperp;s^{(n)}\right)\label{eq:imp:double_strati_int}.
\end{equation}
Schließlich sind die Branchingstrati $S_{\alpha\beta}$ mit
\begin{equation}
  S_{\alpha\beta}:=\frac{1}{\sigma_{\nd}}\sum_{k\in S_a}\sum_{l\in S_b}\sum_{m,n}\frac{\partial^3\sigma_{kl\to mn}\left(x_1,x_2,\pperp;s\right)}{\partial x_1\ \partial x_2\ \partial \pperp}
\end{equation}
als bedingte Wahrscheinlichkeit dafür definiert, dass eine hadronische Wechselwirkung aus dem Stratum $\{\alpha,\beta\}$ stattfindet, gegeben dass eine nicht-diffraktive hadronische Wechselwirkung stattfindet. $\sigma_{\nd}$ ist der totale, nicht-diffraktive Wirkungsquerschnitt und einer der freien Parameter des MPI-Modells. Die einfachen Strati $S_a$ sind in Tabelle \ref{tab:all:strati:strati} dargestellt.

\wip{In Tabelle \eqref{tab:all:strati:strati} sind zwei Varianten angegeben. In der ersten fehlen offensichtlich die Quasivalezquarks. Diese werden zwar vollkommen korrekt in den Remnant-Strukturfunktionen berücksichtigt, werden aber für die eigentliche Generierung der MPI aus technischen Gründen komplett ignoriert. In der Dissertation ist in Kapitel 5.2 ein Vorschlag gemacht, wie Quasivalenzquarks mitgenommen werden können. Diese Umsetzung bedeutet aber einen erheblichen Eingriff in den Quellcode.}


\begin{table}
\begin{center}
\subfloat[]{
\begin{tabular}{ccc}
  Stratum&Name&Partonen\\
  \midrule
  $S_1$&Gluon&$g$\\
  $S_2$&See&$\{q^S:\ \forall q\}$\\
  $S_3$&Valenz-Down&$d^V$\\
  $S_4$&Valenz-Up&$u^V$
\end{tabular}}\qquad
\subfloat[]{
\begin{tabular}{ccc}
  Stratum&Name&Partonen\\
  \midrule
  $S_1$&Gluon&$g$\\
  $S_2$&See&$\{q^S:\ \forall q\}$\\
  $S_3$&Valenz-Down&$d^V$\\
  $S_4$&Valenz-Up&$u^V$\\
  $S_5$&Quasivalenz&$\{q^Q:\ \forall q\}$
\end{tabular}}
\end{center}
\caption{\label{tab:all:strati:strati}(a): Strati zur Berechnung der nächsten Skala. (b) Strati für die Wichtungsfaktoren in \eqref{eq:all:rem:sumrule}}
\end{table}
In \CompRef{muli\_interactions}{valid\_processes} ist für jedes Feynmandiagramm in der fünften Komponente valid\_processes(5,:) die Nummer des Stratums eingetragen, zu dem das Diagramm gehört.

\mip{Die einfachen Strati $S_\alpha$ heißen im Quellcode pdf\_int\_kind. Sie sind in \CompRef{muli\_interactions}{pdf\_int\_kind\_gluon} und folgende definiert. Entsprechend sind die doppelten Strati $\{\alpha,\beta\}$ in \CompRef{muli\_interactions}{double\_pdf\_kinds} hinterlegt.}

Wir bestimmen die nächste Skala des Stratums $\{\alpha,\beta\}$ über
\begin{equation}
  \label{eq:all:genpt-a}
  \widehat{p}_{\perp,a,b}^{(n)}=\mathcal{S}_{\alpha\beta}^{-1}\left(\ \cdot\ ;s^{(n)}\right)\left(\zeta_{\alpha\beta}^{(n)}\right)
\end{equation}
mit
\begin{equation}
  \label{eq:all:genpt-b}
  \zeta_{\alpha\beta}^{(n)}:=
  \frac{
    \ln(z_{\alpha\beta}^{(n)})
  }
  {W_a^{(n)}W_b^{(n)}}
  +
  \mathcal{S}_{\alpha\beta}\left(\pperp^{(n-1)};s^{(n)}\right)
  =
  \mathcal{S}_{\alpha\beta}\left(\pperp^{(n)};s^{(n)}\right)
\end{equation}
und
\begin{equation}
  \label{eq:all:genpt-c}
  z_{\alpha\beta}^{(n)} \in (0,1],\quad \text{zufällig und gleichverteilt.}
\end{equation}
Durch einsetzen von $s^{(n)}$ in $\mathcal{S}_{\alpha\beta}$ erhalten wir eine einstellige, umkehrbare Funktion $\mathcal{S}_{\alpha\beta}\left(\ \cdot\ ;s^{(n)}\right)$. Somit ist $\mathcal{S}_{\alpha\beta}^{-1}\left(\ \cdot\ ;s^{(n)}\right)\left(\zeta_{\alpha\beta}^{(n)}\right)$ eben diese Umkehrfunktion, ausgewertet bei $\zeta_{\alpha\beta}^{(n)}$.

Der größte Wert von $\widehat{p}_{\perp,a,b}^{(n)}$ unter allen Strati ist die neue Skala $\widehat{p}_{\perp}^{(n)}$, das Stratum mit der größten Skala ist das neue Stratum $\{\alpha^{(n)},\beta^{(n)}\}$

\wip{Die Stammstrati in \eqref{eq:all:strati_root_def} hängen offensichtlich von der aktuellen hadronischen invarianten Masse $s^{(n)}$ ab. Derzeit werden die $\mathcal{S}_{\alpha\beta}$ aber als eindimensionale Funktionen in \CompRef{muli\_type}{dsigma} ohne $s$-Abhängigkeit gespeichert. Geht man zu einer zweidimensionalen Dastellung über, dann kommt man zu den Performance- und Speicherproblemen, die in der Dissertation in Kapitel 5.2 beschrieben sind. Dynamische Werte von $s^{(n)}$ sind also nicht durch einen trivialen Patch implementierbar. Das Problem wird teilweise dadurch entschärft, dass die Skala später auf die invariante Masse normiert wird. Quotienten $\pperp/s$ werden also korrekt behandelt, nur durch Faktoren von $s$ ohne $\pperp$ werden die Matrixelemente inkonsistent.}
\section{Importance Sampling}\label{sec:all:alg:imp}
Das Stratum $\{\alpha,\beta\}$ sowie die Skala $\pperp$ liegen fest. Wie auch in der Dissertation unterdrücken wir hier die Indizes $^{(n)}$, da hier nur Werte aus der aktuellen Iteration vorkommen. Wir generieren die hyperbolischen Impulsanteile $h_1,h_2$, indem wir die Gleichung
\begin{equation}
  zs\overline{S}_{ab}\left(\pperp(h_3,s^{(n)}),s^{(n)}\right)\overset{?}{<}H_{ab}\left(h_1,h_2,h_3,s^{(n)}\right)
\end{equation}
mit
\begin{equation}
  h_1,h_2,z\in(0,1],\quad \text{zufällig und gleichverteilt}
\end{equation}
solange mit neu generierten $h_1,h_2,z$ auswerten, bis sie erfüllt ist. $H$ ist eine regularisierte Form der divergenten Branchingstrati $S$, mit
\begin{equation}
  H_{ab}\left(h_1,h_2,h_3,s^{(n)}\right)=S_{ab}\left(x_1(h_1,h_2),x_2(h_1,h_2),\pperp(h_1,h_2,s^{(n)}),s^{(n)}\right)\left(\frac{\partial h_1}{\partial x_1}\frac{\partial h_2}{\partial x_2}-\frac{\partial h_1}{\partial x_2}\frac{\partial h_2}{\partial x_1}\right)
\end{equation}
und
\begin{align}\label{eq:all:imp:trafo}
  h_1&:=\frac{x_1x_2 - h_3}{(1 - h_3)^{1/4}}\\
  h_2&:=\frac{1+(x_2^2 - x_1^2)^{1/3}}{2}\\
  h_3&:=\frac{4\pperp}{\widehat{s}^{(n)}}\\
  x_1&=\sqrt{\sqrt{(h_1^4(1-h_3)+h_3)^2+(4(h_2-1/2)^3)^2}-4(h_2-1/2)^3}\\
  x_2&=\sqrt{\sqrt{(h_1^4(1-h_3)+h_3)^2+(4(h_2-1/2)^3)^2}+4(h_2-1/2)^3}\\
  \pperp&=\frac{h_3\widehat{s}^{(n)}}{4}.
\end{align}
$\overline{S}_{ab}$ ist der Mittelwert von $H_{ab}$, gemittelt über $h_1$ und $h_2$. Mit dem willkürlichen reellen Faktor $s$ wird $s\overline{S}_{ab}$ damit zu einer, von $\pperp$ abhängigen, Majorante von $H_{ab}$.

Um $H$ noch weiter zu glätten, wird in \CompRef{muli\_type}{samples} eine Einteilung des $\{h_1,h_2,h_3\}$-Einheits\-quaders und Unterquader abgelegt. Dabei wird der Quader zuerst so in $h_3$-Richtung in endlich viele Scheiben geschnitten, dass das Integral über $H$ in jeder Scheibe etwa gleich ist. Anschließend wird jede Scheibe simultan in $h_1$ und $h_2$ so in Unterquader zerlegt, dass das Integral über jedes dieser Unterquader etwa gleich groß ist und die Varianz in jedem Unterquader klein wird. Jeder Unterquader hat einen Index $q_i$ und einen Flächeninhalt in der $h_1-h_2$-Ebene von $a_i$.

Weiterhin werden alle Feynmandiagramme, die in dem Stratus $\{\alpha,\beta\}$ enthalten sind, mit einem Wicht\-ungs\-faktor $d_j$ versehen. Die tatsächliche Vorgehensweise zur Generierung der Impulse und der Flavor ist dann wie folgt:

\begin{enumerate}
  \item Es wird ein Diagramm mit der Wahrscheinlichkeit $W_j/\sum_k W_k$ gewählt.
  \item Es wird zufällig und gleichverteilt ein Quader $q_i$ mit der Fläche $a_i$ aus derjenigen Scheibe gewählt, die $h_3$ enthält.
  \item Es werden zufällig und gleichverteilt reelle Zahlen $h_1,h_2$ und $z$ aus dem Einheitsintervall gewählt
  \item Es wird 
    \begin{equation}\label{eq:all:importance}
      zsW_j\overline{S}_{ab}\left(\pperp(h_3,s^{(n)}),s^{(n)}\right)\overset{?}{<}a_iH_{ab}\left(h_1,h_2,h_3,s^{(n)}\right)
    \end{equation}
    ausgewertet und bei 1 begonnen, bis die Ungleichung erfüllt ist.
  \item
    Aus $h_1$ und $h_2$ ergeben sich die Impulsanteile $x_1$ und $x_2$, aus dem Diagramm $j$ ergeben sich die Flavor $a,b,c$ und $d$.
\end{enumerate}
\section{Remnants}
Abweichend von der Dissertation (Gl. 4.49) werden hier fünf verschiedene Wichtungsfaktoren für die vier Strati $\{$Gluon, See, Valenz-down, Valenz-up, Quasivalenz$\}$ zugelassen:
\begin{equation}
  \begin{split}
    1=\\
    W_G^{(n)}\int_{\xi_{\min}^{(n)}}^{1}\der \xi^{(n)}f_g(\xi^{(n)},Q^2)+\\
    +W_S^{(n)}\sum_q\int_{\xi_{\min}^{(n)}}^{1}\der \xi^{(n)}f_{q^{S}}(\xi^{(n)},Q^2)+\\
    +\sum_qW_{q^V}^{(n)}\frac{N_{q^V}^{(n)}}{N_{q^V}^{(0)}}\int_{\xi_{\min}^{(n)}}^{1}\der \xi^{(n)}f_q^v(\xi^{(n)},Q^2)+\\
    +W_Q^{(n)}\sum_q\int_{\xi_{\min}^{(n)}}^{1}\der \xi^{(n)}f_q^Q(\xi^{(n)},Q^2)\label{eq:all:rem:sumrule}
  \end{split}
\end{equation}
Da wir nur eine Gleichung für vier Wichtungsfaktoren haben, müssen wir weitere Beziehungen festlegen. Durch den Parameter \CompRef{muli\_remnant}{remnant\_weight\_model} wird entschieden, welche Wichtungsfaktoren auf Eins gesetzt werden. Die jeweils anderen werden gleich gesetzt. Für das Quadrupel $[W_G,W_S,W_{d^V},W_{u^V},W_Q]$ erhalten wir:

\begin{table}
  \begin{center}
    \begin{tabular}{cc}
      remnant\_weight\_model&$[W_G,W_S,W_d,W_u,W_Q]$\\
      \midrule
      $0$&$[1,\ 1,\ 1,\ 1,\ 1\ ]$\\
      $1$&$[w,w,w,w,w]$\\
      $2$&$[w,w,1,\ 1,\ 1\ ]$\\
      $3$&$[1,\ 1,\ w,w,w]$\\
      $4$&$[1,\ w,1,\ 1,\ w]$
    \end{tabular}
  \end{center}
  \caption{\label{tab:all:rem:weight_models}remnant\_weight\_models}
\end{table}
In \eqref{eq:all:rem:sumrule} eingesetzt kann $w$ eindeutig bestimmt werden.

\mip{Für remnant\_weight\_model=0 ist \eqref{eq:all:rem:sumrule} nicht lösbar, es ist also streng genommen kein gültiges Wichtungsmodell. Stattdessen wird dadurch die Gewichtung deaktiviert.}

\section{Programmfluss}
Die einzeilnen Methoden sind ausführlich in \ModuleRef{muli} beschrieben. Wir geben hier nur eine kurze Übersicht an:
\begin{itemize}
\item initialize

  Der Monte-Carlo-Generator von MulI und die Datenstruktur der Proton-Remnants werden initialisiert.
\item apply\_initial\_interaction

  Eine von WHIZARD generierte harte Wechselwirkung wird an MulI übergeben. Die Remnants werden entsprechend angepasst.
\item generate\_gev2\_pt2
  
  Mittels \eqref{eq:all:genpt-a} wird eine Skala $\widehat{p}_{\perp,a,b}^{(n)}$ und ein Stratum $\{\alpha^{(n)},\beta^{(n)}\}$ generiert.
\item generate\_partons
  
  Mittels \eqref{eq:all:importance} werden die Impulsanteile $x_1$ und $x_2$ und die Flavor $a,b,c$ und $d$ generiert. Außerdem wird eine interne Darstellung der Farbflüsse generiert.
\item get\_correlations

Die interne Darstellung der Farbflüsse wird in der vom shower\_interface gewünschten Form von Farbkorrelationen ausgegeben.
\item replace\_parton

  Der ISR-Algorithmus hat ein Branching eines aktiven Showerteilchens generiert. Dieses Showerteilchen ist fortan kein aktives Teilchen mehr, sondern ein inneres Teilchen der perturbativen Wechselwirkung. Entsprechend muss es durch replace\_parton in der Beschreibung des Remnants durch das neue aktive Teilchen ersetzt werden. Siehe Abbildung \ref{fig:all:flow:isr}
  \begin{figure}
    \begin{center}
      \includegraphics{diagrams-4.mps}\includegraphics{diagrams-5.mps}\includegraphics{diagrams-6.mps}
      \caption{\label{fig:all:flow:isr}Ersetzung des aktiven ISR-Partons durch den ISR-Algorithmus. Links: Durch die härteste WW wurde ein Parton aus dem Hadron entfert. Dieses Parton ist ein "`aktives"' Parton, da ISR-Branchings für dieses Teilchen generiert werden. Alle Teilchen, die jemals generiert werden, bekommen eine eindeutige Nummer, hier die Nummer 1. Die Eigenschaften des Remants und des Teichens \#1 müssen in der Summe die Eigenschaften des Protons ergeben. Rechts: Durch den ISR-Algorithmus wurde ein Branching der Teilches \#1 erzeugt. Das Mutterparton hat die Nummer 2 bekommen, das andere Tochterparton hat die Nummer 3 bekommen. Jetzt müssen die Eigenschaften des Remnants und des Teilchens \#2 in der Summe die Eigenschaften des Protons ergeben. In diesem Sinne müssen wir das Teilchen \#1 wieder "`zurücklegen"' und das Teilchen \#2 "`herausnehmen"'.}
    \end{center}
  \end{figure}
\item restart

  Es werden einige interne Variablen zurückgesetzt, wie z.B. das "finished"-Flag. Außerdem werden die Remnants zurückgesetzt.
\item finalize

  Der Monte-Carlo-Generator von MulI, die vorgenerierten Wirkungsquerschnitte und die Remnants werden deallociert.
\end{itemize}
