\Module{muli\_momentum}
%\begin{figure}
%  \centering{\includegraphics{uml-module-tree-6.mps}}
%  \caption{\label{fig:\ThisModule:Types}Klassendiagramm des Moduls \ThisModule}
%\end{figure}
\section{Abhängigkeiten}
\use{muli\_basic}
\section{Derived Types}
\TypeDef{transversal\_momentum\_type}
Dieser Datentyp abstrahiert den Entwicklungsparameter $\pperp$. Intern wird $\pperp$ durch die Komponente \LocalVar{momentum} dargestellt. Dieser enthält die fünf Einträge $\big[s, \sqrt{\pperp}, \pperp, \sqrt{4*\pperp/s}, 4*\pperp/s\big]$. Für jeden Eintrag werden get und set Methoden bereitgestellt. Wenn statt eines Werts von $\pperp$ eine Instanz vom Typ \LocalVar{transversal\_momentum\_type} übergeben wird, werden Fehler durch falsche Einheiten vermieden.

Nach MulI-Konvention wird die Einheit immer vor den Namen der Variable gestellt, also GeV\_scale für $\sqrt{\pperp}$ und GeV2\_scale für $\pperp$ usw. Hier wird bei dimensionslosen Größen $\sim\sqrt{\pperp}$ das Prefix unit und bei dimensionslosen Größen $\sim\pperp$ das prefix unit2 vorangestellt, damit immer klar ist, was gemeint ist.

\LocalVar{MaxScale} ist der größtmögliche Wert für $\pperp$, mit $\pperp^{\max}=s/2$.

\wip{Die invariante Masse $s$ wird hier \LocalVar{initial\_cme} genannt. Letzteres ist die invariante Masse des Proton-Proton-Systems vor der ersten (WHIZARD) Wechselwirkung. Das spiegelt die Tatsache wider, dass dynamische Energieen der Remnants noch nicht implementiert sind. Dieser Datentyp ist der richtige Ort, um die aktuelle CME zu speichern. Da alle anderen Module auf die get-Methoden dieses Moduls zurückgreifen, sollte es ausreichen, die interne Darstellung hier anzupassen.}

\begin{Verbatim}
implicit none
  type,\Extends{serializable\_class}::transversal_momentum_type
     private
     real(kind=drk),dimension(0:4)::\TC{momentum}=[0D0,0D0,0D0,0D0,0D0]
   contains
     \OverridesDeclaration{serializable\_class}
     procedure,public::\TbpDec{write\_to\_marker}{transversal\_momentum\_write\_to\_marker}
     procedure,public::\TbpDec{read\_from\_marker}{transversal\_momentum\_read\_from\_marker}
     procedure,public::\TbpDec{print\_to\_unit}{transversal\_momentum\_print\_to\_unit}
     procedure,public,nopass::\TbpDec{get\_type}{transversal\_momentum\_get\_type}
     \OriginalDeclaration{transversal\_momentum\_type}
     procedure,public::\TbpDec{get\_gev\_initial\_cme}{transversal\_momentum\_get\_gev\_initial\_cme}
     procedure,public::\TbpDec{get\_gev\_max\_scale}{transversal\_momentum\_get\_gev\_max\_scale}
     procedure,public::\TbpDec{get\_gev2\_max\_scale}{transversal\_momentum\_get\_gev2\_max\_scale}
     procedure,public::\TbpDec{get\_gev\_scale}{transversal\_momentum\_get\_gev\_scale}
     procedure,public::\TbpDec{get\_gev2\_scale}{transversal\_momentum\_get\_gev2\_scale}
     procedure,public::\TbpDec{get\_unit\_scale}{transversal\_momentum\_get\_unit\_scale}
     procedure,public::\TbpDec{get\_unit2\_scale}{transversal\_momentum\_get\_unit2\_scale}
     procedure,public::\TbpDec{set\_gev\_initial\_cme}{transversal\_momentum\_set\_gev\_initial\_cme}
     procedure,public::\TbpDec{set\_gev\_max\_scale}{transversal\_momentum\_set\_gev\_max\_scale}
     procedure,public::\TbpDec{set\_gev2\_max\_scale}{transversal\_momentum\_set\_gev2\_max\_scale}
     procedure,public::\TbpDec{set\_gev\_scale}{transversal\_momentum\_set\_gev\_scale}
     procedure,public::\TbpDec{set\_gev2\_scale}{transversal\_momentum\_set\_gev2\_scale}
     procedure,public::\TbpDec{set\_unit\_scale}{transversal\_momentum\_set\_unit\_scale}
     procedure,public::\TbpDec{set\_unit2\_scale}{transversal\_momentum\_set\_unit2\_scale}
     procedure,public::\TbpDecS{transversal\_momentum\_initialize}
     generic,public::\TbpDec{initialize}{transversal\_momentum\_initialize}
  end type transversal_momentum_type
\end{Verbatim}
\TypeDef{qcd\_2\_2\_class}
Abstrakte Klasse, die eine QCD-2$\rightarrow$2-Wechselwirkung abstrahiert. \TypeRef{pp\_remnant\_type} greift auf Eigenschaften einer solchen Wechselwirkung zurück, allerdings werden die Methoden in dem Modul \ModuleRef{muli} implementiert, auf das \ModuleRef{muli\_remnant} keinen Zugriff hat. Zwar könnte man dieses Problem durch eine andere Hierarchie von Modulen lösen, aber ich nehme an, dass dieses Problem wieder auftaucht, wenn die Remnants als WHIZARD-Strukturfunktionen implementiert werden. Deswegen habe ich diese Lösung gewählt.
\begin{Verbatim}
  type,Extends{transversal\_momentum\_type},abstract::qcd_2_2_class
   contains
     procedure(qcd_get_int),deferred::\TbpDef{get\_process\_id}
     procedure(qcd_get_int),deferred::\TbpDef{get\_integrand\_id}
     procedure(qcd_get_int),deferred::\TbpDef{get\_diagram\_kind}
     procedure(qcd_get_int_4),deferred::\TbpDef{get\_lha\_flavors}
     procedure(qcd_get_int_4),deferred::\TbpDef{get\_pdg\_flavors}
     procedure(qcd_get_int_by_int),deferred::\TbpDef{get\_parton\_id}
     procedure(qcd_get_int_2),deferred::\TbpDef{get\_parton\_kinds}
     procedure(qcd_get_int_2),deferred::\TbpDef{get\_pdf\_int\_kinds}
     procedure(qcd_get_drk),deferred::\TbpDef{get\_momentum\_boost}
     procedure(qcd_get_drk_2),deferred::\TbpDef{get\_remnant\_momentum\_fractions}
     procedure(qcd_get_drk_2),deferred::\TbpDef{get\_total\_momentum\_fractions}
  end type qcd_2_2_class
\end{Verbatim}
\section{Interfaces}
\begin{Verbatim}
    abstract interface
     subroutine qcd_none(this)
       import qcd_2_2_class
       class(qcd_2_2_class),target,intent(in)::this
     end subroutine qcd_none
     elemental function qcd_get_drk(this)
       use muli_basic, only: drk
       import qcd_2_2_class
       class(qcd_2_2_class),intent(in)::this
       real(kind=drk)::qcd_get_drk
     end function qcd_get_drk
     pure function qcd_get_drk_2(this)
       use muli_basic, only: drk
       import qcd_2_2_class
       class(qcd_2_2_class),intent(in)::this
       real(kind=drk),dimension(2)::qcd_get_drk_2
     end function qcd_get_drk_2
     pure function qcd_get_drk_3(this)
       use muli_basic, only: drk
       import qcd_2_2_class
       class(qcd_2_2_class),intent(in)::this
       real(kind=drk),dimension(3)::qcd_get_drk_3
     end function qcd_get_drk_3
     elemental function qcd_get_int(this)
       use muli_basic, only: drk
       import qcd_2_2_class
       class(qcd_2_2_class),intent(in)::this
       integer::qcd_get_int
     end function qcd_get_int
     elemental function qcd_get_int_by_int(this,n)
       use muli_basic, only: drk
       import qcd_2_2_class
       class(qcd_2_2_class),intent(in)::this
       integer,intent(in)::n
       integer::qcd_get_int_by_int
     end function qcd_get_int_by_int
     pure function qcd_get_int_2(this)
       use muli_basic, only: drk
       import qcd_2_2_class
       class(qcd_2_2_class),intent(in)::this
       integer,dimension(2)::qcd_get_int_2
     end function qcd_get_int_2
     pure function qcd_get_int_4(this)
       use muli_basic, only: drk
       import qcd_2_2_class
       class(qcd_2_2_class),intent(in)::this
       integer,dimension(4)::qcd_get_int_4
     end function qcd_get_int_4
  end interface
\end{Verbatim}
\Methods
\MethodsFor{transversal\_momentum\_type}
\OverridesSection{serializable\_class}
\TbpImp{transversal\_momentum\_write\_to\_marker}
\begin{Verbatim}
  subroutine transversal_momentum_write_to_marker(this,marker,status)
    class(transversal_momentum_type),intent(in)::this
    class(marker_type),intent(inout)::marker
    integer(kind=dik),intent(out)::status
    call marker%mark_begin("transversal_momentum_type")
    call marker%mark("gev_momenta",this%momentum(0:1))
    call marker%mark_end("transversal_momentum_type")
  end subroutine transversal_momentum_write_to_marker
\end{Verbatim}
\TbpImp{transversal\_momentum\_read\_from\_marker}
\begin{Verbatim}
  subroutine transversal_momentum_read_from_marker(this,marker,status)
    class(transversal_momentum_type),intent(out)::this
    class(marker_type),intent(inout)::marker
    integer(kind=dik),intent(out)::status
    call marker%pick_begin("transversal_momentum_type",status=status)
    call marker%pick("gev_momenta",this%momentum(0:1),status)
    this%momentum(2:4)=[&
         this%momentum(1)**2,&
         this%momentum(1)/this%momentum(0),&
         (this%momentum(1)/this%momentum(0))**2]
    call marker%pick_end("transversal_momentum_type",status=status)
  end subroutine transversal_momentum_read_from_marker
\end{Verbatim}
\TbpImp{transversal\_momentum\_print\_to\_unit}
\begin{Verbatim}
  subroutine transversal_momentum_print_to_unit(this,unit,parents,components,peers)
    class(transversal_momentum_type),intent(in)::this
    integer,intent(in)::unit
    integer(kind=dik),intent(in)::parents,components,peers
    write(unit,'("Components of transversal_momentum_type:")')
    write(unit,fmt='("Actual energy scale:")')
    write(unit,fmt='("Max scale (MeV)   :",E20.10)')this%momentum(0)
    write(unit,fmt='("Scale (MeV)       :",E20.10)')this%momentum(1)
    write(unit,fmt='("Scale^2 (MeV^2)   :",E20.10)')this%momentum(2)
    write(unit,fmt='("Scale normalized  :",E20.10)')this%momentum(3)
    write(unit,fmt='("Scale^2 normalized:",E20.10)')this%momentum(4)
  end subroutine transversal_momentum_print_to_unit
  \end{Verbatim}
\TbpImp{transversal\_momentum\_get\_type}
\begin{Verbatim}
  pure subroutine transversal_momentum_get_type(type)
    character(:),allocatable,intent(out)::type
    allocate(type,source="transversal_momentum_type")
  end subroutine transversal_momentum_get_type
\end{Verbatim}
\OriginalSection{transversal\_momentum\_type}
\TbpImp{transversal\_momentum\_get\_gev\_initial\_cme}
\begin{Verbatim}
  elemental function transversal_momentum_get_gev_initial_cme(this) result(scale)
    class(transversal_momentum_type),intent(in)::this
    real(kind=drk)::scale
    scale=this%momentum(0)*2D0
  end function transversal_momentum_get_gev_initial_cme
\end{Verbatim}
\TbpImp{transversal\_momentum\_get\_gev\_max\_scale}

\TbpImp{transversal\_momentum\_get\_gev\_max\_scale}
\begin{Verbatim}
  elemental function transversal_momentum_get_gev_max_scale(this) result(scale)
    class(transversal_momentum_type),intent(in)::this
    real(kind=drk)::scale
    scale=this%momentum(0)
  end function transversal_momentum_get_gev_max_scale
\end{Verbatim}

\TbpImp{transversal\_momentum\_get\_gev2\_max\_scale}
\begin{Verbatim}
  elemental function transversal_momentum_get_gev2_max_scale(this) result(scale)
    class(transversal_momentum_type),intent(in)::this
    real(kind=drk)::scale
    scale=this%momentum(0)**2
  end function transversal_momentum_get_gev2_max_scale
\end{Verbatim}

\TbpImp{transversal\_momentum\_get\_gev\_scale}
\begin{Verbatim}
  elemental function transversal_momentum_get_gev_scale(this) result(scale)
    class(transversal_momentum_type),intent(in)::this
    real(kind=drk)::scale
    scale=this%momentum(1)
  end function transversal_momentum_get_gev_scale
\end{Verbatim}

\TbpImp{transversal\_momentum\_get\_gev2\_scale}
\begin{Verbatim}
  elemental function transversal_momentum_get_gev2_scale(this) result(scale)
    class(transversal_momentum_type),intent(in)::this
    real(kind=drk)::scale
    scale=this%momentum(2)
  end function transversal_momentum_get_gev2_scale
\end{Verbatim}

\TbpImp{transversal\_momentum\_get\_unit\_scale}
\begin{Verbatim}
  elemental function transversal_momentum_get_unit_scale(this) result(scale)
    class(transversal_momentum_type),intent(in)::this
    real(kind=drk)::scale
    scale=this%momentum(3)
  end function transversal_momentum_get_unit_scale
\end{Verbatim}

\TbpImp{transversal\_momentum\_get\_unit2\_scale}
\begin{Verbatim}
  elemental function transversal_momentum_get_unit2_scale(this) result(scale)
    class(transversal_momentum_type),intent(in)::this
    real(kind=drk)::scale
    scale=this%momentum(4)
  end function transversal_momentum_get_unit2_scale
\end{Verbatim}

\TbpImp{transversal\_momentum\_set\_gev\_initial\_cme}
\begin{Verbatim}
  subroutine transversal_momentum_set_gev_initial_cme(this,new_gev_initial_cme)
    class(transversal_momentum_type),intent(inout)::this
    real(kind=drk),intent(in) :: new_gev_initial_cme
    this%momentum(0) = new_gev_initial_cme/2D0
    this%momentum(3) = this%momentum(1)/this%momentum(0)
    this%momentum(4) = this%momentum(3)**2
  end subroutine transversal_momentum_set_gev_initial_cme
\end{Verbatim}

\TbpImp{transversal\_momentum\_set\_gev\_max\_scale}
\begin{Verbatim}
  subroutine transversal_momentum_set_gev_max_scale(this,new_gev_max_scale)
    class(transversal_momentum_type),intent(inout)::this
    real(kind=drk),intent(in) :: new_gev_max_scale
    this%momentum(0) = new_gev_max_scale
    this%momentum(3) = this%momentum(1)/this%momentum(0)
    this%momentum(4) = this%momentum(3)**2
  end subroutine transversal_momentum_set_gev_max_scale
\end{Verbatim}

\TbpImp{transversal\_momentum\_set\_gev2\_max\_scale}
\begin{Verbatim}
  subroutine transversal_momentum_set_gev2_max_scale(this,new_gev2_max_scale)
    class(transversal_momentum_type),intent(inout)::this
    real(kind=drk),intent(in) :: new_gev2_max_scale
    this%momentum(0) = sqrt(new_gev2_max_scale)
    this%momentum(3) = this%momentum(1)/this%momentum(0)
    this%momentum(4) = this%momentum(3)**2
  end subroutine transversal_momentum_set_gev2_max_scale
\end{Verbatim}

\TbpImp{transversal\_momentum\_set\_gev\_scale}
\begin{Verbatim}
  subroutine transversal_momentum_set_gev_scale(this,new_gev_scale)
    class(transversal_momentum_type),intent(inout)::this
    real(kind=drk),intent(in) :: new_gev_scale
    this%momentum(1) = new_gev_scale
    this%momentum(2) = new_gev_scale**2
    this%momentum(3) = new_gev_scale/this%momentum(0)
    this%momentum(4) = this%momentum(3)**2
  end subroutine transversal_momentum_set_gev_scale
\end{Verbatim}

\TbpImp{transversal\_momentum\_set\_gev2\_scale}
\begin{Verbatim}
  subroutine transversal_momentum_set_gev2_scale(this,new_gev2_scale)
    class(transversal_momentum_type),intent(inout)::this
    real(kind=drk),intent(in) :: new_gev2_scale
    this%momentum(1) = sqrt(new_gev2_scale)
    this%momentum(2) = new_gev2_scale
    this%momentum(3) = this%momentum(1)/this%momentum(0)
    this%momentum(4) = this%momentum(3)**2
  end subroutine transversal_momentum_set_gev2_scale
\end{Verbatim}

\TbpImp{transversal\_momentum\_set\_unit\_scale}
\begin{Verbatim}
  subroutine transversal_momentum_set_unit_scale(this,new_unit_scale)
    class(transversal_momentum_type),intent(inout)::this
    real(kind=drk),intent(in) :: new_unit_scale
    this%momentum(1) = new_unit_scale*this%momentum(0)
    this%momentum(2) = this%momentum(1)**2
    this%momentum(3) = new_unit_scale
    this%momentum(4) = this%momentum(3)**2
  end subroutine transversal_momentum_set_unit_scale
\end{Verbatim}

\TbpImp{transversal\_momentum\_set\_unit2\_scale}
\begin{Verbatim}
  subroutine transversal_momentum_set_unit2_scale(this,new_unit2_scale)
    class(transversal_momentum_type),intent(inout)::this
    real(kind=drk),intent(in) :: new_unit2_scale
    this%momentum(3) = sqrt(new_unit2_scale)
    this%momentum(4) = new_unit2_scale
    this%momentum(1) = this%momentum(3)*this%momentum(0)
    this%momentum(2) = this%momentum(1)**2
  end subroutine transversal_momentum_set_unit2_scale
\end{Verbatim}

\TbpImp{transversal\_momentum\_initialize}
\begin{Verbatim}
  subroutine transversal_momentum_initialize(this,gev2_s)
    class(transversal_momentum_type),intent(out)::this
    real(kind=drk),intent(in)::gev2_s
    real(kind=drk)::gev_s
    gev_s=sqrt(gev2_s)
    this%momentum=[gev_s/2D0,gev_s/2D0,gev2_s/4D0,1D0,1D0]
  end subroutine transversal_momentum_initialize
\end{Verbatim}
