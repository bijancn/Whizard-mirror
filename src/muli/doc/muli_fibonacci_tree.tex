\Module{muli\_fibonacci\_tree}
%\begin{figure}
%  \centering{\includegraphics{uml-module-tree-11.mps}}
%  \caption{\label{fig:\ThisModule:Types}Klassendiagramm des Moduls \ThisModule}
%\end{figure}
\section{Abhängigkeiten}
\use{muli\_basic}
\section{Parameter}
\begin{Verbatim}
  character(*),parameter,private :: no_par = "edge={\textbackslash}noparent"
  character(*),parameter,private :: no_ret = "edge={\textbackslash}noreturn"
  character(*),parameter,private :: no_kid = "edge={\textbackslash}nochild"
  character(*),parameter,private :: le_kid = "edge={\textbackslash}childofleave"
\end{Verbatim}

\section{Derived Types}
\TypeDef{fibonacci\_node\_type}
\begin{Verbatim}
  type,\Extends{measurable\_class} :: fibonacci_node_type
!     private
     class(\TypeRef{fibonacci\_node\_type}), pointer :: \TC{up} => null()
     class(\TypeRef{measurable\_class}), pointer :: \TC{down} => null()
     class(\TypeRef{fibonacci\_node\_type}), pointer :: \TC{left} => null()
     class(\TypeRef{fibonacci\_node\_type}), pointer :: \TC{right} => null()
     integer :: depth = 0
   contains
     ! overridden serializable_class procedures
     procedure::\TbpDec{write\_to\_marker}{fibonacci\_node\_write\_to\_marker}
     procedure::\TbpDec{read\_from\_marker}{fibonacci\_node\_read\_from\_marker}
     procedure::\TbpDec{read\_target\_from\_marker}{fibonacci\_node\_read\_target\_from\_marker}
     procedure::\TbpDec{print\_to\_unit}{fibonacci\_node\_print\_to\_unit}
     procedure,nopass::\TbpDec{get\_type}{fibonacci\_node\_get\_type}
     procedure::\TbpDec{deserialize\_from\_marker}{fibonacci\_node\_deserialize\_from\_marker}
     ! overridden measurable_class procedures
     procedure::\TbpDec{measure}{fibonacci\_node\_measure}
     ! init/final
     procedure,public ::\TbpDec{deallocate\_tree}{fibonacci\_node\_deallocate\_tree}
     procedure,public ::\TbpDec{deallocate\_all}{fibonacci\_node\_deallocate\_all}
!     interface
     procedure,public ::\TbpDec{get\_depth}{fibonacci\_node\_get\_depth}
     procedure,public ::\TbpDec{count\_leaves}{fibonacci\_node\_count\_leaves}
! public tests
     procedure,public,nopass ::\TbpDec{is\_leave}{fibonacci\_node\_is\_leave}
     procedure,public,nopass ::\TbpDec{is\_root}{fibonacci\_node\_is\_root}
     procedure,public,nopass ::\TbpDec{is\_inner}{fibonacci\_node\_is\_inner}
! print methods
     procedure,public ::\TbpDec{write\_association}{fibonacci\_node\_write\_association}
     procedure,public ::\TbpDec{write\_contents}{fibonacci\_node\_write\_contents}
     procedure,public ::\TbpDec{write\_values}{fibonacci\_node\_write\_values}
     procedure,public ::\TbpDec{write\_leaves}{fibonacci\_node\_write\_leaves}
     !procedure,public ::\TbpDec{write}{fibonacci\_node\_write\_contents}
! write methods
     procedure,public ::\TbpDec{write\_pstricks}{fibonacci\_node\_write\_pstricks}
! elaborated functions
     procedure,public ::\TbpDec{copy\_node}{fibonacci\_node\_copy\_node}
     procedure,public ::\TbpDec{find\_root}{fibonacci\_node\_find\_root}
     procedure,public ::\TbpDec{find\_leftmost}{fibonacci\_node\_find\_leftmost}
     procedure,public ::\TbpDec{find\_rightmost}{fibonacci\_node\_find\_rightmost}
     procedure,public ::\TbpDec{find}{fibonacci\_node\_find}
     procedure,public ::\TbpDec{find\_left\_leave}{fibonacci\_node\_find\_left\_leave}
     procedure,public ::\TbpDec{find\_right\_leave}{fibonacci\_node\_find\_right\_leave}
     procedure,public ::\TbpDec{apply\_to\_leaves}{fibonacci\_node\_apply\_to\_leaves}
     procedure,public ::\TbpDec{apply\_to\_leaves\_rl}{fibonacci\_node\_apply\_to\_leaves\_rl}
! private procedures: these are unsafe!
     procedure ::\TbpDec{set\_depth}{fibonacci\_node\_set\_depth}
     procedure ::\TbpDec{append\_left}{fibonacci\_node\_append\_left}
     procedure ::\TbpDec{append\_right}{fibonacci\_node\_append\_right}
     procedure ::\TbpDec{replace}{fibonacci\_node\_replace}
     procedure ::\TbpDec{swap}{fibonacci\_node\_swap\_nodes}
     procedure ::\TbpDec{flip}{fibonacci\_node\_flip\_children}
     procedure ::\TbpDec{rip}{fibonacci\_node\_rip}
     procedure ::\TbpDec{remove\_and\_keep\_parent}{fibonacci\_node\_remove\_and\_keep\_parent}
     procedure ::\TbpDec{remove\_and\_keep\_twin}{fibonacci\_node\_remove\_and\_keep\_twin}
     procedure ::\TbpDec{rotate\_left}{fibonacci\_node\_rotate\_left}
     procedure ::\TbpDec{rotate\_right}{fibonacci\_node\_rotate\_right}
     procedure ::\TbpDec{rotate}{fibonacci\_node\_rotate}
     procedure ::\TbpDec{balance\_node}{fibonacci\_node\_balance\_node}
     procedure ::\TbpDec{update\_depth\_save}{fibonacci\_node\_update\_depth\_save}
     procedure ::\TbpDec{update\_depth\_unsave}{fibonacci\_node\_update\_depth\_unsave}
     procedure ::\TbpDec{repair}{fibonacci\_node\_repair}
! tests: these are save when type is fibonacci_node_type and else unsafe.
     procedure ::\TbpDec{is\_left\_short}{fibonacci\_node\_is\_left\_short}
     procedure ::\TbpDec{is\_right\_short}{fibonacci\_node\_is\_right\_short}
     procedure ::\TbpDec{is\_unbalanced}{fibonacci\_node\_is\_unbalanced}
     procedure ::\TbpDec{is\_left\_too\_short}{fibonacci\_node\_is\_left\_too\_short}
     procedure ::\TbpDec{is\_right\_too\_short}{fibonacci\_node\_is\_right\_too\_short}
     procedure ::\TbpDec{is\_too\_unbalanced}{fibonacci\_node\_is\_too\_unbalanced}
     procedure ::\TbpDec{is\_left\_child}{fibonacci\_node\_is\_left\_child}
     procedure ::\TbpDec{is\_right\_child}{fibonacci\_node\_is\_right\_child}
  end type fibonacci_node_type
\end{Verbatim}
\TypeDef{fibonacci\_leave\_type}
\begin{Verbatim}
  type,\Extends{fibonacci\_node\_type} :: fibonacci_leave_type
!     class(\TypeRef{measurable\_class}),pointer :: \TC{content}
  contains
     ! overridden serializable_class procedures
     procedure::\TbpDec{print\_to\_unit}{fibonacci\_leave\_print\_to\_unit}
     procedure,nopass::\TbpDec{get\_type}{fibonacci\_leave\_get\_type}
     procedure,public ::\TbpDec{deallocate\_all}{fibonacci\_leave\_deallocate\_all}
     ! new procedures
     procedure,public ::\TbpDec{pick}{fibonacci\_leave\_pick}
     procedure,public ::\TbpDec{get\_left}{fibonacci\_leave\_get\_left}
     procedure,public ::\TbpDec{get\_right}{fibonacci\_leave\_get\_right}
     procedure,public ::\TbpDec{write\_pstricks}{fibonacci\_leave\_write\_pstricks}
     procedure,public ::\TbpDec{copy\_content}{fibonacci\_leave\_copy\_content}
     procedure,public ::\TbpDec{set\_content}{fibonacci\_leave\_set\_content}
     procedure,public ::\TbpDec{get\_content}{fibonacci\_leave\_get\_content}
     procedure,public,nopass ::\TbpDec{is\_inner}{fibonacci\_leave\_is\_inner}
     procedure,public,nopass ::\TbpDec{is\_leave}{fibonacci\_leave\_is\_leave}
     procedure ::\TbpDec{insert\_leave\_by\_node}{fibonacci\_leave\_insert\_leave\_by\_node}
     procedure ::\TbpDec{is\_left\_short}{fibonacci\_leave\_is\_left\_short}
     procedure ::\TbpDec{is\_right\_short}{fibonacci\_leave\_is\_right\_short}
     procedure ::\TbpDec{is\_unbalanced}{fibonacci\_leave\_is\_unbalanced}
     procedure ::\TbpDec{is\_left\_too\_short}{fibonacci\_leave\_is\_left\_too\_short}
     procedure ::\TbpDec{is\_right\_too\_short}{fibonacci\_leave\_is\_right\_too\_short}
     procedure ::\TbpDec{is\_too\_unbalanced}{fibonacci\_leave\_is\_too\_unbalanced}
  end type fibonacci_leave_type
\end{Verbatim}
\TypeDef{fibonacci\_root\_type}
\begin{Verbatim}
  type,\Extends{fibonacci\_node\_type} :: fibonacci_root_type
     logical::\TC{is\_valid\_c}=.false.
     class(\TypeRef{fibonacci\_leave\_type}),pointer ::\TC{leftmost}=>null()
     class(\TypeRef{fibonacci\_leave\_type}),pointer ::\TC{rightmost}=>null()
  contains
     ! overridden serializable_class procedures
     procedure::\TbpDec{write\_to\_marker}{fibonacci\_root\_write\_to\_marker}
     procedure::\TbpDec{read\_target\_from\_marker}{fibonacci\_root\_read\_target\_from\_marker}
     procedure::\TbpDec{print\_to\_unit}{fibonacci\_root\_print\_to\_unit}
     procedure,nopass::\TbpDec{get\_type}{fibonacci\_root\_get\_type}
     ! new procedures
     procedure::\TbpDec{get\_leftmost}{fibonacci\_root\_get\_leftmost}
     procedure::\TbpDec{get\_rightmost}{fibonacci\_root\_get\_rightmost}
! public tests
     procedure,public,nopass ::\TbpDec{is\_root}{fibonacci\_root\_is\_root}
     procedure,public,nopass ::\TbpDec{is\_inner}{fibonacci\_root\_is\_inner}
     procedure,public ::\TbpDec{is\_valid}{fibonacci\_root\_is\_valid}
     procedure,public ::\TbpDec{count\_leaves}{fibonacci\_root\_count\_leaves}
     procedure,public ::\TbpDec{write\_pstricks}{fibonacci\_root\_write\_pstricks}
     procedure,public ::\TbpDec{copy\_root}{fibonacci\_root\_copy\_root}
     procedure,public ::\TbpDec{push\_by\_content}{fibonacci\_root\_push\_by\_content}
     procedure,public ::\TbpDec{push\_by\_leave}{fibonacci\_root\_push\_by\_leave}
     procedure,public ::\TbpDec{pop\_left}{fibonacci\_root\_pop\_left}
     procedure,public ::\TbpDec{pop\_right}{fibonacci\_root\_pop\_right}
     procedure,public ::\TbpDec{merge}{fibonacci\_root\_merge}
     procedure,public ::\TbpDec{set\_leftmost}{fibonacci\_root\_set\_leftmost}
     procedure,public ::\TbpDec{set\_rightmost}{fibonacci\_root\_set\_rightmost}
     procedure,public ::\TbpDec{init\_by\_leave}{fibonacci\_root\_init\_by\_leave}
     procedure,public ::\TbpDec{init\_by\_content}{fibonacci\_root\_init\_by\_content}
     procedure,public ::\TbpDec{reset}{fibonacci\_root\_reset}
     ! init/final
     procedure,public ::\TbpDec{deallocate\_tree}{fibonacci\_root\_deallocate\_tree}
     procedure,public ::\TbpDec{deallocate\_all}{fibonacci\_root\_deallocate\_all}
     procedure ::\TbpDec{is\_left\_child}{fibonacci\_root\_is\_left\_child}
     procedure ::\TbpDec{is\_right\_child}{fibonacci\_root\_is\_right\_child}
  end type fibonacci_root_type
\end{Verbatim}
\TypeDef{fibonacci\_stub\_type}
\begin{Verbatim}
   type,\Extends{fibonacci\_root\_type} :: fibonacci_stub_type
   contains
     ! overridden serializable_class procedures
     procedure,nopass::\TbpDec{get\_type}{fibonacci\_stub\_get\_type}
     ! overridden fibonacci_root_type procedures
     procedure,public ::\TbpDec{push\_by\_content}{fibonacci\_stub\_push\_by\_content}
     procedure,public ::\TbpDec{push\_by\_leave}{fibonacci\_stub\_push\_by\_leave}
     procedure,public ::\TbpDec{pop\_left}{fibonacci\_stub\_pop\_left}
     procedure,public ::\TbpDec{pop\_right}{fibonacci\_stub\_pop\_right}
  end type fibonacci_stub_type
\end{Verbatim}
\TypeDef{fibonacci\_leave\_list\_type}
\begin{Verbatim}
  type fibonacci_leave_list_type
     class(\TypeRef{fibonacci\_leave\_type}),pointer ::\TC{leave}=>null()
     class(\TypeRef{fibonacci\_leave\_list\_type}),pointer :: \TC{next} => null()
  end type fibonacci_leave_list_type
\end{Verbatim}
\Methods
\MethodsFor{fibonacci\_node\_type}
\OverridesSection{serializable\_class}

\TbpImp{fibonacci\_node\_write\_to\_marker}
\begin{Verbatim}
 recursive  subroutine fibonacci_node_write_to_marker(this,marker,status)
    class(fibonacci_node_type), intent(in) :: this
    class(marker_type),intent(inout)::marker
    integer(kind=dik),intent(out)::status
! local variables
    class(serializable_class),pointer::ser
    call marker%mark_begin("fibonacci_node_type")
    ser=>this%left
    call marker%mark_pointer("left",ser)
    ser=>this%right
    call marker%mark_pointer("right",ser)
    ser=>this%xxxx
    call marker%mark_pointer("down",ser)
    call marker%mark_end("fibonacci_node_type")
  end subroutine fibonacci_node_write_to_marker
\end{Verbatim}

\TbpImp{fibonacci\_node\_read\_from\_marker}
\begin{Verbatim}
  recursive subroutine fibonacci_node_read_from_marker (this,marker,status)
    class(fibonacci_node_type), intent(out) :: this
    class(marker_type),intent(inout)::marker
    integer(kind=dik),intent(out)::status
    print *,"fibonacci_node_read_from_marker: You cannot deserialize a list with this subroutine."
    print *,"Use fibonacci_node_read_target_from_marker instead."    
  end subroutine fibonacci_node_read_from_marker
\end{Verbatim}

\TbpImp{fibonacci\_node\_read\_target\_from\_marker}
\begin{Verbatim}
recursive subroutine fibonacci_node_read_target_from_marker(this,marker,status)
    class(fibonacci_node_type),target,intent(out) :: this
    class(marker_type),intent(inout)::marker
    integer(kind=dik),intent(out)::status
! local variables
    class(serializable_class),pointer::ser
    call marker%pick_begin("fibonacci_node_type",status=status)
    call marker%pick_pointer("left",ser)
    if(status==0)then
       select type(ser)
       class is (fibonacci_node_type)
          this%left=>ser
          this%left%up=>this
       end select
    end if
    call marker%pick_pointer("right",ser)
    if(status==0)then
       select type(ser)
       class is (fibonacci_node_type)
          this%right=>ser
          this%right%up=>this
       end select
    end if
    call marker%pick_pointer("down",ser)
    if(status==0)then
       select type(ser)
       class is (measurable_class)
          this%xxxx=>ser
       end select
    end if
    call marker%pick_end("fibonacci_node_type",status)
  end subroutine fibonacci_node_read_target_from_marker
\end{Verbatim}

\TbpImp{fibonacci\_node\_get\_type}
\begin{Verbatim}
  pure subroutine fibonacci_node_get_type(type)
    character(:),allocatable,intent(out)::type
    allocate(type,source="fibonacci_node_type")
  end subroutine fibonacci_node_get_type
\end{Verbatim}
  
\TbpImp{fibonacci\_node\_deserialize\_from\_marker}
\begin{Verbatim}
  subroutine fibonacci_node_deserialize_from_marker(this,name,marker)
    class(fibonacci_node_type),intent(out)::this
    character(*),intent(in)::name
    class(marker_type),intent(inout)::marker
    class(serializable_class),pointer::ser
    allocate(fibonacci_leave_type::ser)
    call marker%push_reference(ser)
    allocate(fibonacci_node_type::ser)
    call marker%push_reference(ser)
    call serializable_deserialize_from_marker(this,name,marker)
    call marker%pop_reference(ser)
    deallocate(ser)
    call marker%pop_reference(ser)
    deallocate(ser)    
  end subroutine fibonacci_node_deserialize_from_marker
\end{Verbatim}
    
\TbpImp{fibonacci\_node\_print\_to\_unit}
\begin{Verbatim}
  recursive subroutine fibonacci_node_print_to_unit(this,unit,parents,components,peers)
    class(fibonacci_node_type),intent(in)::this
    integer,intent(in)::unit
    integer(kind=dik),intent(in)::parents,components,peers
    class(serializable_class),pointer::ser
    write(unit,'("Components of fibonacci_node_type:")')
    write(unit,'("Depth:   ",I22)')this%depth
    write(unit,'("Value:   ",E23.16)')this%measure()
    ser=>this%up
    call serialize_print_comp_pointer(ser,unit,parents,-one,-one,"Up:     ")
    ser=>this%left
    call serialize_print_peer_pointer(ser,unit,parents,components,peers,"Left:   ")
    ser=>this%right
    call serialize_print_peer_pointer(ser,unit,parents,components,peers,"Right:  ")
  end subroutine fibonacci_node_print_to_unit
\end{Verbatim}

\TbpImp{fibonacci\_node\_measure}
\begin{Verbatim}
  elemental function fibonacci_node_measure(this)
    class(fibonacci_node_type),intent(in)::this
    real(kind=double)::fibonacci_node_measure
    fibonacci_node_measure=this%down%measure()
  end function fibonacci_node_measure
\end{Verbatim}

  ! init/final

\TbpImp{fibonacci\_node\_deallocate\_tree}
\begin{Verbatim}
  recursive subroutine fibonacci_node_deallocate_tree(this)
    class(fibonacci_node_type),intent(inout) :: this
    if (associated(this%left)) then
       call this%left%deallocate_tree()
       deallocate(this%left)
    end if
    if (associated(this%right)) then
       call this%right%deallocate_tree()
       deallocate(this%right)
    end if
    call this%set_depth(0)
  end subroutine fibonacci_node_deallocate_tree
\end{Verbatim}

\TbpImp{fibonacci\_node\_deallocate\_all}
\begin{Verbatim}
  recursive subroutine fibonacci_node_deallocate_all(this)
    class(fibonacci_node_type),intent(inout) :: this
    if (associated(this%left)) then
       call this%left%deallocate_all()
       deallocate(this%left)
    end if
    if (associated(this%right)) then
       call this%right%deallocate_all()
       deallocate(this%right)
    end if
    call this%set_depth(0)
  end subroutine fibonacci_node_deallocate_all
\end{Verbatim}

\TbpImp{fibonacci\_node\_set\_depth}
\begin{Verbatim}
  subroutine fibonacci_node_set_depth(this,depth)
    class(fibonacci_node_type),intent(inout) :: this
    integer,intent(in) :: depth
    this%depth=depth
  end subroutine fibonacci_node_set_depth
\end{Verbatim}

\TbpImp{fibonacci\_node\_get\_depth}
\begin{Verbatim}
  elemental function fibonacci_node_get_depth(this)
    class(fibonacci_node_type),intent(in) :: this
    integer :: fibonacci_node_get_depth
    fibonacci_node_get_depth = this%depth
  end function fibonacci_node_get_depth
\end{Verbatim}

\TbpImp{fibonacci\_node\_is\_leave}
\begin{Verbatim}
  elemental function fibonacci_node_is_leave()
    logical :: fibonacci_node_is_leave
    fibonacci_node_is_leave = .false.
  end function fibonacci_node_is_leave
\end{Verbatim}

\TbpImp{fibonacci\_node\_is\_root}
\begin{Verbatim}
  elemental function fibonacci_node_is_root()
    logical :: fibonacci_node_is_root
    fibonacci_node_is_root = .false.
  end function fibonacci_node_is_root
\end{Verbatim}

\TbpImp{fibonacci\_node\_is\_inner}
\begin{Verbatim}
  elemental function fibonacci_node_is_inner()
    logical :: fibonacci_node_is_inner
    fibonacci_node_is_inner = .true.
  end function fibonacci_node_is_inner
\end{Verbatim}

\TbpImp{fibonacci\_node\_write\_leaves}
\begin{Verbatim}
  subroutine fibonacci_node_write_leaves(this,unit)
    class(fibonacci_node_type),intent(in),target :: this
    integer,intent(in),optional :: unit
    call this%apply_to_leaves(fibonacci_leave_write,unit)
  end subroutine fibonacci_node_write_leaves
\end{Verbatim}

\TbpImp{fibonacci\_node\_write\_contents}
\begin{Verbatim}
  subroutine fibonacci_node_write_contents(this,unit)
    class(fibonacci_node_type),intent(in),target :: this
    integer,intent(in),optional :: unit
    call this%apply_to_leaves(fibonacci_leave_write_content,unit)
  end subroutine fibonacci_node_write_contents
\end{Verbatim}

\TbpImp{fibonacci\_node\_write\_values}
\begin{Verbatim}
  subroutine fibonacci_node_write_values(this,unit)
    class(fibonacci_node_type),intent(in),target :: this
    integer,intent(in),optional :: unit
    call this%apply_to_leaves(fibonacci_leave_write_value,unit)
  end subroutine fibonacci_node_write_values
\end{Verbatim}

\TbpImp{fibonacci\_node\_write\_association}
\begin{Verbatim}
  subroutine fibonacci_node_write_association(this,that)
    class(fibonacci_node_type),intent(in),target :: this
    class(fibonacci_node_type),intent(in),target :: that
    if (associated(that%left,this)) then
       write(*,'("this is left child of that")')
    end if
    if (associated(that%right,this)) then
       write(*,'("this is right child of that")')
    end if
    if (associated(that%up,this)) then
       write(*,'("this is parent of that")')
    end if    
    if (associated(this%left,that)) then
       write(*,'("that is left child of this")')
    end if
    if (associated(this%right,that)) then
       write(*,'("that is right child of this")')
    end if
    if (associated(this%up,that)) then
       write(*,'("that is parent of this")')
    end if
  end subroutine fibonacci_node_write_association
\end{Verbatim}
  
\TbpImp{fibonacci\_node\_write\_pstricks}
\begin{Verbatim}
  recursive subroutine fibonacci_node_write_pstricks(this,unitnr)
    class(fibonacci_node_type),intent(in),target :: this
    integer,intent(in) :: unitnr
    if (associated(this%up)) then
       if (associated(this%up%left,this).neqv.(associated(this%up%right,this))) then
          write(unitnr,'("{\textbackslash}begin{psTree}{{\textbackslash}Toval{{\textbackslash}node{",i3,"}{",f9.3,"}}}")')&
            int(this%depth),this%measure()
       else
          write(unitnr,'("{\textbackslash}begin{psTree}{{\textbackslash}Toval[",a,"]{{\textbackslash}node{",i3,"}{",f9.3,"}}}")')&
            no_ret,int(this%depth),this%measure()
       end if
    else
       write(unitnr,'("{\textbackslash}begin{psTree}{{\textbackslash}Toval[",a,"]{{\textbackslash}node{",i3,"}{",f9.3,"}}}")')&
         no_par,int(this%depth),this%measure()
    end if
    if (associated(this%left)) then
       call this%left%write_pstricks(unitnr)
    else
       write(unitnr,'("{\textbackslash}Tr[edge=brokenline]{}")')
    end if
    if (associated(this%right)) then
       call this%right%write_pstricks(unitnr)
    else
       write(unitnr,'("{\textbackslash}Tr[edge=brokenline]{}")')
    end if
    write(unitnr,'("{\textbackslash}end{psTree}")')
  end subroutine fibonacci_node_write_pstricks
\end{Verbatim}

\TbpImp{fibonacci\_node\_copy\_node}
\begin{Verbatim}
  subroutine fibonacci_node_copy_node(this,primitive)
    class(fibonacci_node_type),intent(out) :: this
    class(fibonacci_node_type),intent(in) :: primitive
    this%up => primitive%up
    this%left => primitive%left
    this%right => primitive%right
    this%depth = primitive%depth
    this%down=> primitive%down
  end subroutine fibonacci_node_copy_node
\end{Verbatim}

\TbpImp{fibonacci\_node\_find\_root}
\begin{Verbatim}
  subroutine fibonacci_node_find_root(this,root)
    class(fibonacci_node_type),intent(in),target  :: this
    class(fibonacci_root_type),pointer,intent(out) :: root
    class(fibonacci_node_type),pointer :: node
    node=>this
    do while(associated(node%up))
       node=>node%up
    end do
    select type (node)
    class is (fibonacci_root_type)
       root=>node
    class default
       nullify(root)
       print *,"fibonacci_node_find_root: root is not type compatible to&
       & fibonacci_root_type. Retured NULL()."
    end select
  end subroutine fibonacci_node_find_root
\end{Verbatim}

\TbpImp{fibonacci\_node\_find\_leftmost}
\begin{Verbatim}
  subroutine fibonacci_node_find_leftmost(this,leave)
    class(fibonacci_node_type),intent(in), target  :: this
    class(fibonacci_leave_type),pointer,intent(out) :: leave
    class(fibonacci_node_type), pointer :: node
    node=>this
    do while(associated(node%left))
       node=>node%left
    end do
    select type (node)
    class is (fibonacci_leave_type)
       leave => node
    class default
       leave => null()
    end select
  end subroutine fibonacci_node_find_leftmost
\end{Verbatim}

\TbpImp{fibonacci\_node\_find\_rightmost}
\begin{Verbatim}
  subroutine fibonacci_node_find_rightmost(this,leave)
    class(fibonacci_node_type),intent(in), target  :: this
    class(fibonacci_leave_type),pointer,intent(out) :: leave
    class(fibonacci_node_type), pointer :: node
    node=>this
    do while(associated(node%right))
       node=>node%right
    end do
    select type (node)
    class is (fibonacci_leave_type)
       leave => node
    class default
       leave => null()
    end select
  end subroutine fibonacci_node_find_rightmost
\end{Verbatim}

\TbpImp{fibonacci\_node\_find}
\begin{Verbatim}
  subroutine fibonacci_node_find(this,value,leave)
    class(fibonacci_node_type),intent(in),target  :: this
    real(kind=double),intent(in) :: value
    class(fibonacci_leave_type),pointer,intent(out) :: leave
    class(fibonacci_node_type), pointer :: node
    node=>this
    do
       if (node>=value) then
          if (associated(node%left)) then
             node=>node%left
          else
             print *,"fibonacci_node_find: broken tree!"
             leave => null()
             return
          end if
       else
          if (associated(node%right)) then
             node=>node%right
          else
             print *,"fibonacci_node_find: broken tree!"
             leave => null()
             return
          end if
       end if
       select type (node)
       class is (fibonacci_leave_type)
          leave => node
          exit
       end select
    end do
  end subroutine fibonacci_node_find
\end{Verbatim}

\TbpImp{fibonacci\_node\_find\_left\_leave}
\begin{Verbatim}
  subroutine fibonacci_node_find_left_leave(this,leave)
    class(fibonacci_node_type),intent(in),target  :: this
    class(fibonacci_node_type),pointer  :: node
    class(fibonacci_leave_type),pointer,intent(out)  :: leave
    nullify(leave)
    node=>this
    do while (associated(node%up))
       if (associated(node%up%right,node)) then
          node=>node%up%left
          do while (associated(node%right))
             node=>node%right
          end do
          select type (node)
          class is (fibonacci_leave_type)
          leave=>node
          end select
          exit
       end if
       node=>node%up
    end do
  end subroutine fibonacci_node_find_left_leave
\end{Verbatim}

\TbpImp{fibonacci\_node\_find\_right\_leave}
\begin{Verbatim}
  subroutine fibonacci_node_find_right_leave(this,leave)
    class(fibonacci_node_type),intent(in),target  :: this
    class(fibonacci_node_type),pointer  :: node
    class(fibonacci_leave_type),pointer,intent(out)  :: leave
    nullify(leave)
    node=>this
    do while (associated(node%up))
       if (associated(node%up%left,node)) then
          node=>node%up%right
          do while (associated(node%left))
             node=>node%left
          end do
          select type (node)
          class is (fibonacci_leave_type)
          leave=>node
          end select
          exit
       end if
       node=>node%up
    end do
  end subroutine fibonacci_node_find_right_leave
\end{Verbatim}

\TbpImp{fibonacci\_node\_replace}
\begin{Verbatim}
  subroutine fibonacci_node_replace(this,old_node)
    class(fibonacci_node_type),intent(inout),target :: this
    class(fibonacci_node_type),target :: old_node
    if (associated(old_node%up)) then
       if (old_node%is_left_child()) then
          old_node%up%left => this
       else
          if (old_node%is_right_child()) then
             old_node%up%right => this
          end if
       end if
       this%up => old_node%up
    else
       nullify(this%up)
    end if
  end subroutine fibonacci_node_replace
\end{Verbatim}

\TbpImp{fibonacci\_node\_swap\_nodes}
\begin{Verbatim}
  subroutine fibonacci_node_swap_nodes(left,right)
    class(fibonacci_node_type),target,intent(inout) :: left,right
    class(fibonacci_node_type),pointer :: left_left,right_right
    class(measurable_class),pointer::down
    ! swap branches
    left_left  =>left%left
    right_right=>right%right
    left%left  =>right%right
    right%right=>left_left
    ! repair up components
    right_right%up=>left
    left_left%up  =>right
    ! repair down components
          down =>  left%down
     left%down => right%down
    right%down =>       down
  end subroutine fibonacci_node_swap_nodes
\end{Verbatim}

\TbpImp{fibonacci\_node\_swap\_nodes}
\begin{Verbatim}
!  subroutine fibonacci_node_swap_nodes(this,that)
!    class(fibonacci_node_type),target :: this
!    class(fibonacci_node_type),pointer,intent(in) :: that
!    class(fibonacci_node_type),pointer :: par_i,par_a
!    par_i => this%up
!    par_a => that%up
!    if (associated(par_i%left,this)) then
!       par_i%left => that
!    else
!       par_i%right => that
!    end if
!    if (associated(par_a%left,that)) then
!       par_a%left => this
!    else
!       par_a%right => this
!    end if
!    this%up => par_a
!    that%up => par_i
!  end subroutine fibonacci_node_swap_nodes
\end{Verbatim}
  
\TbpImp{fibonacci\_node\_flip\_children}
\begin{Verbatim}
  subroutine fibonacci_node_flip_children(this)
    class(fibonacci_node_type),intent(inout) :: this
    class(fibonacci_node_type),pointer :: child
    child => this%left
    this%left=>this%right
    this%right => child
  end subroutine fibonacci_node_flip_children
\end{Verbatim}

\TbpImp{fibonacci\_node\_rip}
\begin{Verbatim}
  subroutine fibonacci_node_rip(this)
    class(fibonacci_node_type),intent(inout),target :: this
    if (this%is_left_child()) then
       nullify(this%up%left)
    end if
    if (this%is_right_child()) then
       nullify(this%up%right)
    end if
    nullify(this%up)
  end subroutine fibonacci_node_rip
\end{Verbatim}
 
\TbpImp{fibonacci\_node\_remove\_and\_keep\_twin}
\begin{Verbatim}
  subroutine fibonacci_node_remove_and_keep_twin(this,twin)
    class(fibonacci_node_type),intent(inout),target  :: this
    class(fibonacci_node_type),intent(out),pointer :: twin
    class(fibonacci_node_type),pointer :: pa
    if (.not. (this%is_root())) then
       pa=>this%up
       if (.not. pa%is_root()) then
          if (this%is_left_child()) then         
             twin => pa%right
          else
             twin => pa%left
          end if
          if (pa%is_left_child()) then
             pa%up%left => twin
          else
             pa%up%right => twin
          end if
       end if
       twin%up => pa%up
       if(associated(this%right))then
          this%right%left=>this%left
       end if
       if(associated(this%left))then
          this%left%right=>this%right
       end if
       nullify(this%left)
       nullify(this%right)
       nullify(this%up)
       deallocate(pa)
    end if
  end subroutine fibonacci_node_remove_and_keep_twin
\end{Verbatim}

\TbpImp{fibonacci\_node\_remove\_and\_keep\_parent}
\begin{Verbatim}
  subroutine fibonacci_node_remove_and_keep_parent(this,pa)
    class(fibonacci_node_type),intent(inout),target  :: this
    class(fibonacci_node_type),intent(out),pointer :: pa
    class(fibonacci_node_type),pointer :: twin
    if (.not. (this%is_root())) then
       pa=>this%up
       if (this%is_left_child()) then         
          twin => pa%right
       else
          twin => pa%left
       end if
       twin%up=>pa%up
       if (associated(twin%left)) then
          twin%left%up => pa
       end if
       if (associated(twin%right)) then
          twin%right%up => pa
       end if
       call pa%copy_node(twin)
       select type(pa)
       class is (fibonacci_root_type)
          call pa%set_leftmost()
          call pa%set_rightmost()
       end select
       if(associated(this%right))then
          this%right%left=>this%left
       end if
       if(associated(this%left))then
          this%left%right=>this%right
       end if
       nullify(this%left)
       nullify(this%right)
       nullify(this%up)
       deallocate(twin)
    else
       pa=>this
    end if
  end subroutine fibonacci_node_remove_and_keep_parent
\end{Verbatim}

\TbpImp{fibonacci\_leave\_pick}
\begin{Verbatim}
  subroutine fibonacci_leave_pick(this)
    class(fibonacci_leave_type),target,intent(inout) :: this
    class(fibonacci_node_type),pointer :: other
    class(fibonacci_root_type),pointer :: root
!    call this%up%print_parents()
    call this%find_root(root)
    if(associated(this%up,root))then
       if(this%up%depth<2)then
          print *,"fibonacci_leave_pick: Cannot pick leave. &
          &Tree must have at least three leaves."
          return
       else
          call this%remove_and_keep_parent(other)
          call other%repair()
       end if
    else
       call this%remove_and_keep_twin(other)
       call other%up%repair()
    end if
    if(associated(root%leftmost,this))call root%set_leftmost()
    if(associated(root%rightmost,this))call root%set_rightmost()
  end subroutine fibonacci_leave_pick
\end{Verbatim}

\TbpImp{fibonacci\_node\_append\_left}
\begin{Verbatim}
  subroutine fibonacci_node_append_left(this,new_branch)
    class(fibonacci_node_type),target :: this
    class(fibonacci_node_type),target :: new_branch
    this%left => new_branch
    new_branch%up => this
  end subroutine fibonacci_node_append_left
\end{Verbatim}
  
\TbpImp{fibonacci\_node\_append\_right}
\begin{Verbatim}
  subroutine fibonacci_node_append_right(this,new_branch)
    class(fibonacci_node_type),intent(inout),target :: this
    class(fibonacci_node_type),target :: new_branch
    this%right => new_branch
    new_branch%up => this
  end subroutine fibonacci_node_append_right
\end{Verbatim}

\TbpImp{fibonacci\_node\_rotate\_left}
\begin{Verbatim}
  subroutine fibonacci_node_rotate_left(this)
    class(fibonacci_node_type),intent(inout),target  :: this
    call this%swap(this%right)
    call this%right%flip()
    call this%right%update_depth_unsave()
    call this%flip()
!    value = this%value
!    this%value = this%left%value
!    this%left%value = value
  end subroutine fibonacci_node_rotate_left
\end{Verbatim}

\TbpImp{fibonacci\_node\_rotate\_right}
\begin{Verbatim}
  subroutine fibonacci_node_rotate_right(this)
    class(fibonacci_node_type),intent(inout),target  :: this
    call this%left%swap(this)
    call this%left%flip()
    call this%left%update_depth_unsave()
    call this%flip()
!    value = this%value
!    this%value = this%right%value
!    this%right%value = value
  end subroutine fibonacci_node_rotate_right
\end{Verbatim}

\TbpImp{fibonacci\_node\_rotate}
\begin{Verbatim}
  subroutine fibonacci_node_rotate(this)
    class(fibonacci_node_type),intent(inout),target  :: this
    if (this%is_left_short()) then
       call this%rotate_left()
    else
       if (this%is_right_short()) then
          call this%rotate_right()
       end if
    end if
  end subroutine fibonacci_node_rotate
\end{Verbatim}

\TbpImp{fibonacci\_node\_balance\_node}
\begin{Verbatim}
  subroutine fibonacci_node_balance_node(this,changed)
    class(fibonacci_node_type),intent(inout),target  :: this
    logical,intent(out) :: changed
    changed=.false.
    if (this%is_left_too_short()) then
       if (this%right%is_right_short()) then
          call this%right%rotate_right
       end if
       call this%rotate_left()
       changed=.true.
    else
       if (this%is_right_too_short()) then
          if (this%left%is_left_short()) then
             call this%left%rotate_left
          end if
          call this%rotate_right()
          changed=.true.
       end if
    end if
  end subroutine fibonacci_node_balance_node
\end{Verbatim}

\TbpImp{fibonacci\_node\_update\_depth\_unsave}
\begin{Verbatim}
  subroutine fibonacci_node_update_depth_unsave(this)
    class(fibonacci_node_type),intent(inout)  :: this
    this%depth=max(this%left%depth+1,this%right%depth+1)
  end subroutine fibonacci_node_update_depth_unsave
\end{Verbatim}

\TbpImp{fibonacci\_node\_update\_depth\_save}
\begin{Verbatim}
  subroutine fibonacci_node_update_depth_save(this,updated)
    class(fibonacci_node_type),intent(inout)  :: this
    logical,intent(out) :: updated
    integer :: left,right,new_depth
    if (associated(this%left)) then
       left=this%left%depth+1
    else
       left=-1
    end if
    if (associated(this%right)) then
       right=this%right%depth+1
    else
       right=-1
    end if
    new_depth=max(left,right)
    if (this%depth == new_depth) then
       updated = .false.
    else
       this%depth=new_depth
       updated = .true.
    end if
  end subroutine fibonacci_node_update_depth_save
\end{Verbatim}

\TbpImp{fibonacci\_node\_repair}
\begin{Verbatim}
  subroutine fibonacci_node_repair(this)
    class(fibonacci_node_type),intent(inout),target :: this
    class(fibonacci_node_type),pointer:: node
    logical :: new_depth,new_balance
    new_depth = .true.
    node=>this
    do while((new_depth .or. new_balance) .and. (associated(node)))
       call node%balance_node(new_balance)
       call node%update_depth_save(new_depth)
       node=>node%up
    end do
  end subroutine fibonacci_node_repair
\end{Verbatim}

\TbpImp{fibonacci\_node\_is\_left\_short}
\begin{Verbatim}
  elemental logical function fibonacci_node_is_left_short(this)
    class(fibonacci_node_type),intent(in) :: this
    fibonacci_node_is_left_short = (this%left%depth<this%right%depth)
  end function fibonacci_node_is_left_short
\end{Verbatim}

\TbpImp{fibonacci\_node\_is\_right\_short}
\begin{Verbatim}
  elemental logical function fibonacci_node_is_right_short(this)
    class(fibonacci_node_type),intent(in) :: this
    fibonacci_node_is_right_short = (this%right%depth<this%left%depth)
  end function fibonacci_node_is_right_short
\end{Verbatim}

\TbpImp{fibonacci\_node\_is\_unbalanced}
\begin{Verbatim}
  elemental logical function fibonacci_node_is_unbalanced(this)
    class(fibonacci_node_type),intent(in) :: this
    fibonacci_node_is_unbalanced = (this%is_left_short() .or. this%is_right_short())
  end function fibonacci_node_is_unbalanced
\end{Verbatim}

\TbpImp{fibonacci\_node\_is\_left\_too\_short}
\begin{Verbatim}
  elemental logical function fibonacci_node_is_left_too_short(this)
    class(fibonacci_node_type),intent(in) :: this
    fibonacci_node_is_left_too_short = (this%left%depth+1<this%right%depth)
  end function fibonacci_node_is_left_too_short
\end{Verbatim}

\TbpImp{fibonacci\_node\_is\_right\_too\_short}
\begin{Verbatim}
  elemental logical function fibonacci_node_is_right_too_short(this)
    class(fibonacci_node_type),intent(in) :: this
    fibonacci_node_is_right_too_short = (this%right%depth+1<this%left%depth)
  end function fibonacci_node_is_right_too_short
\end{Verbatim}

\TbpImp{fibonacci\_node\_is\_too\_unbalanced}
\begin{Verbatim}
  elemental logical function fibonacci_node_is_too_unbalanced(this)
    class(fibonacci_node_type),intent(in) :: this
    fibonacci_node_is_too_unbalanced = (this%is_left_too_short() .or. this%is_right_too_short())
  end function fibonacci_node_is_too_unbalanced
\end{Verbatim}

\TbpImp{fibonacci\_node\_is\_left\_child}
\begin{Verbatim}
  elemental logical function fibonacci_node_is_left_child(this)
    class(fibonacci_node_type),intent(in),target :: this
    fibonacci_node_is_left_child = associated(this%up%left,this)
  end function fibonacci_node_is_left_child
\end{Verbatim}

\TbpImp{fibonacci\_node\_is\_right\_child}
\begin{Verbatim}
  elemental logical function fibonacci_node_is_right_child(this)
    class(fibonacci_node_type),intent(in),target :: this
    fibonacci_node_is_right_child = associated(this%up%right,this)
  end function fibonacci_node_is_right_child
\end{Verbatim}

\TbpImp{fibonacci\_node\_apply\_to\_leaves}
\begin{Verbatim}
  recursive subroutine fibonacci_node_apply_to_leaves(node,func,unit)
    class(fibonacci_node_type),intent(in),target :: node
    interface
       subroutine func(this,unit)
         import fibonacci_leave_type
         class(fibonacci_leave_type),intent(in),target :: this
         integer,intent(in),optional :: unit
       end subroutine func
    end interface
    integer,intent(in),optional :: unit
    select type (node)
    class is (fibonacci_leave_type)
       call func(node,unit)
    class default 
       call node%left%apply_to_leaves(func,unit)
       call node%right%apply_to_leaves(func,unit)
    end select
  end subroutine fibonacci_node_apply_to_leaves
\end{Verbatim}

\TbpImp{fibonacci\_node\_apply\_to\_leaves\_rl}
\begin{Verbatim}
  recursive subroutine fibonacci_node_apply_to_leaves_RL(node,func,unit)
    class(fibonacci_node_type),intent(in),target :: node
    interface
       subroutine func(this,unit)
         import fibonacci_leave_type
         class(fibonacci_leave_type),intent(in),target :: this
         integer,intent(in),optional :: unit
       end subroutine func
    end interface
    integer,intent(in),optional :: unit
    select type (node)
    class is (fibonacci_leave_type)
       call func(node,unit)
    class default 
       call node%right%apply_to_leaves_rl(func,unit)
       call node%left%apply_to_leaves_rl(func,unit)
    end select
  end subroutine fibonacci_node_apply_to_leaves_RL
\end{Verbatim}
  
\TbpImp{fibonacci\_node\_count\_leaves}
\begin{Verbatim}
  recursive subroutine fibonacci_node_count_leaves(this,n)
    class(fibonacci_node_type),intent(in) :: this
    integer,intent(out) :: n
    integer::n1,n2
    if(associated(this%left).and.associated(this%right)) then
       call fibonacci_node_count_leaves(this%left,n1)
       call fibonacci_node_count_leaves(this%right,n2)
       n=n1+n2
    else
       n=1
    end if
  end subroutine fibonacci_node_count_leaves
\end{Verbatim}

\MethodsFor{fibonacci\_root\_type}
\TbpImp{fibonacci\_root\_write\_to\_marker}
\begin{Verbatim}
  SUBROUTINE fibonacci_root_write_to_marker(this,marker,status)
    CLASS(fibonacci_root_type), INTENT(IN) :: this
    class(marker_type),intent(inout)::marker
    integer(kind=dik),intent(out)::status
!    call marker%mark_begin("FIBONACCI_ROOT_TYPE")
    call fibonacci_node_write_to_marker(this,marker,status)
!    marker%mark_end("FIBONACCI_ROOT_TYPE")
  end SUBROUTINE fibonacci_root_write_to_marker
\end{Verbatim}

\TbpImp{fibonacci\_root\_read\_target\_from\_marker}
\begin{Verbatim}
  SUBROUTINE fibonacci_root_read_target_from_marker(this,marker,status)
    CLASS(fibonacci_root_type),target,INTENT(out) :: this
    class(marker_type),intent(inout)::marker
    integer(kind=dik),intent(out)::status
!    call marker%pick_begin("FIBONACCI_ROOT_TYPE",status)
    call fibonacci_node_read_from_marker(this,marker,status)
    call this%find_leftmost(this%leftmost)
    call this%find_rightmost(this%rightmost)
!    call marker%pick_end("FIBONACCI_ROOT_TYPE",status)
  end SUBROUTINE fibonacci_root_read_target_from_marker
\end{Verbatim}

\TbpImp{fibonacci\_root\_print\_to\_unit}
\begin{Verbatim}
  subroutine fibonacci_root_print_to_unit(this,unit,parents,components,peers)
    class(fibonacci_root_type),intent(in)::this
    integer,intent(in)::unit
    integer(kind=dik),intent(in)::parents,components,peers
    class(serializable_class),pointer::ser
    if(parents>0)call fibonacci_node_print_to_unit(this,unit,parents-1,components,peers)
    write(unit,'("Components of fibonacci_root_type:")')
    ser=>this%leftmost
    call serialize_print_peer_pointer(ser,unit,parents,components,min(peers,one),"Leftmost: ")
    ser=>this%rightmost
    call serialize_print_peer_pointer(ser,unit,parents,components,min(peers,one),"Rightmost:")
  end subroutine fibonacci_root_print_to_unit
\end{Verbatim}

\TbpImp{fibonacci\_root\_get\_type}
\begin{Verbatim}
  pure subroutine fibonacci_root_get_type(type)
    character(:),allocatable,intent(out)::type
    allocate(type,source="fibonacci_root_type")
  end subroutine fibonacci_root_get_type
\end{Verbatim}

\TbpImp{fibonacci\_root\_get\_leftmost}
\begin{Verbatim}
  subroutine fibonacci_root_get_leftmost(this,leftmost)
    class(fibonacci_root_type),intent(in)::this
    class(fibonacci_leave_type),pointer::leftmost
    leftmost=>this%leftmost
  end subroutine fibonacci_root_get_leftmost
\end{Verbatim}

\TbpImp{fibonacci\_root\_get\_rightmost}
\begin{Verbatim}
  subroutine fibonacci_root_get_rightmost(this,rightmost)
    class(fibonacci_root_type),intent(in)::this
    class(fibonacci_leave_type),pointer::rightmost
    rightmost=>this%rightmost
  end subroutine fibonacci_root_get_rightmost
\end{Verbatim}

\TbpImp{fibonacci\_root\_is\_inner}
\begin{Verbatim}
  elemental function fibonacci_root_is_inner()
    logical::fibonacci_root_is_inner
    fibonacci_root_is_inner=.false.
  end function fibonacci_root_is_inner
\end{Verbatim}

\TbpImp{fibonacci\_root\_is\_root}
\begin{Verbatim}
  elemental function fibonacci_root_is_root()
    logical::fibonacci_root_is_root
    fibonacci_root_is_root=.true.
  end function fibonacci_root_is_root
\end{Verbatim}

\TbpImp{fibonacci\_root\_is\_valid}
\begin{Verbatim}
  elemental function fibonacci_root_is_valid(this)
    class(fibonacci_root_type),intent(in) :: this
    logical :: fibonacci_root_is_valid
    fibonacci_root_is_valid=this%is_valid_c
  end function fibonacci_root_is_valid
\end{Verbatim}

\TbpImp{fibonacci\_root\_count\_leaves}
\begin{Verbatim}
  subroutine fibonacci_root_count_leaves(this,n)
    class(fibonacci_root_type),intent(in) :: this
    integer,intent(out) :: n
    n=0
    call fibonacci_node_count_leaves(this,n)
  end subroutine fibonacci_root_count_leaves
\end{Verbatim}

\TbpImp{fibonacci\_root\_write\_pstricks}
\begin{Verbatim}
  subroutine fibonacci_root_write_pstricks(this,unitnr)
    class(fibonacci_root_type),intent(in),target :: this
    integer,intent(in) :: unitnr
    logical :: is_opened
    character :: is_sequential,is_formatted,is_writeable
    print *,"pstricks"
    inquire(unitnr,opened=is_opened,&
         &sequential=is_sequential,formatted=is_formatted,write=is_writeable)
    if (is_opened) then
       if (is_sequential=="Y" .and. is_formatted=="Y" .and. is_writeable=="Y") then
          write(unitnr,'("{\textbackslash}begin{psTree}{{\textbackslash}Toval[linecolor=blue]{{\textbackslash}node{",i3,"}{",f9.3,"}}}")')&
            this%depth,this%measure()
          if (associated(this%leftmost)) then
             call this%leftmost%write_pstricks(unitnr)
          else
             write(unitnr,'("{\textbackslash}Tr[",a,"]{}")') no_kid
          end if
          if (associated(this%left)) then
             call this%left%write_pstricks(unitnr)
          else
             write(unitnr,'("{\textbackslash}Tr[",a,"]{}")') no_kid
          end if
          if (associated(this%right)) then
             call this%right%write_pstricks(unitnr)
          else
             write(unitnr,'("{\textbackslash}Tr[",a,"]{}")') no_kid
          end if
          if (associated(this%rightmost)) then
             call this%rightmost%write_pstricks(unitnr)
          else
             write(unitnr,'("{\textbackslash}Tr[",a,"]{}")') no_kid
          end if
          write(unitnr,'("{\textbackslash}end{psTree}")')
          write(unitnr,'("\textbackslash\textbackslash")')
       else
          print '("fibonacci_node_write_pstricks: Unit ",I2," is not opened properly.")',unitnr
          print '("No output is written to unit.")'
       end if
    else
       print '("fibonacci_node_write_pstricks: Unit ",I2," is not opened.")',unitnr
       print '("No output is written to unit.")' 
    end if
  end subroutine fibonacci_root_write_pstricks
\end{Verbatim}

\TbpImp{fibonacci\_root\_copy\_root}
\begin{Verbatim}
  subroutine fibonacci_root_copy_root(this,primitive)
    class(fibonacci_root_type),intent(out) :: this
    class(fibonacci_root_type),intent(in) :: primitive
    call fibonacci_node_copy_node(this,primitive)
    this%leftmost => primitive%leftmost
    this%rightmost => primitive%rightmost
  end subroutine fibonacci_root_copy_root
\end{Verbatim}

\TbpImp{fibonacci\_root\_push\_by\_content}
\begin{Verbatim}
  subroutine fibonacci_root_push_by_content(this,content)
    class(fibonacci_root_type),target,intent(inout)  :: this
    class(measurable_class),target,intent(in)::content
    class(fibonacci_leave_type),pointer :: node
!    print *,"fibonacci_root_push_by_content: ",content%measure()
    allocate(node)
    node%down=>content
    call this%push_by_leave(node)
  end subroutine fibonacci_root_push_by_content
\end{Verbatim}
  
\TbpImp{fibonacci\_root\_push\_by\_leave}
\begin{Verbatim}
  ! this is a workaround for BUG 44696. This subroutine is a merge of 
  ! fibonacci_tree_push_by_node
  ! fibonacci_node_find
  ! fibonacci_leave_insert_leave_by_node
  subroutine fibonacci_root_push_by_leave(this,new_leave)
    class(fibonacci_root_type),target,intent(inout)   :: this
    class(fibonacci_leave_type),pointer,intent(inout) :: new_leave
    class(fibonacci_leave_type),pointer :: old_leave
    class(fibonacci_node_type), pointer :: node,new_node,leave_c
    if (new_leave<=this%leftmost) then
       old_leave=>this%leftmost
       this%leftmost=>new_leave
       node=>old_leave%up
       call fibonacci_node_spawn&
         (new_node,new_leave,old_leave,old_leave%left,old_leave%right)
       call node%append_left(new_node)
    else
       if (new_leave>this%rightmost) then
          old_leave=>this%rightmost
          this%rightmost=>new_leave
          node=>old_leave%up
          call fibonacci_node_spawn&
            (new_node,old_leave,new_leave,old_leave%left,old_leave%right)
          call node%append_right(new_node)
       else
          node=>this
          do
             if (new_leave<=node) then
                leave_c=>node%left
                select type (leave_c)
                class is (fibonacci_leave_type)
                   if(new_leave<=leave_c)then
                      call fibonacci_node_spawn&
                        (new_node,new_leave,leave_c,leave_c%left,leave_c%right)
                   else
                      call fibonacci_node_spawn&
                        (new_node,leave_c,new_leave,leave_c%left,leave_c%right)
                   end if
                   call node%append_left(new_node)
                   exit
                class default
                   node=>node%left
                end select
             else
                leave_c=>node%right
                select type (leave_c)
                class is (fibonacci_leave_type)          
                   if(new_leave<=leave_c)then
                      call fibonacci_node_spawn&
                        (new_node,new_leave,leave_c,leave_c%left,leave_c%right)
                   else
                      call fibonacci_node_spawn&
                        (new_node,leave_c,new_leave,leave_c%left,leave_c%right)
                   end if
                   call node%append_right(new_node)
                   exit
                class default
                   node=>node%right
                end select
             end if
          end do
       end if
    end if
    call node%repair()
  end subroutine fibonacci_root_push_by_leave
\end{Verbatim}

\TbpImp{fibonacci\_root\_pop\_left}
\begin{Verbatim}
  subroutine fibonacci_root_pop_left(this,leave)
    class(fibonacci_root_type),intent(inout),target  :: this
    class(fibonacci_leave_type),pointer,intent(out) :: leave
    class(fibonacci_node_type),pointer  :: parent,grand
    !write(11,fmt=*)"fibonacci root pop left\\"!PSTRICKS
    !flush(11)!PSTRICKS
    leave => this%leftmost
    if (this%left%depth>=1) then
       parent => leave%up
       grand=>parent%up
       grand%left => parent%right
       parent%right%up=>grand
       deallocate(parent)
       parent=>grand%left
       if (.not.parent%is_leave())then
          parent=>parent%left
       end if
       select type (parent)
       class is (fibonacci_leave_type)
          this%leftmost => parent
       class default
          print *,"fibonacci_root_pop_left: ERROR: leftmost is no leave."
          call parent%print_all()
          STOP
       end select
       !call this%write_pstricks(11)!PSTRICKS
       !flush(11)!PSTRICKS
       !write(11,fmt=*)"fibonacci node repair\\"!PSTRICKS
       !flush(11)!PSTRICKS
       call grand%repair()
    else
       if (this%left%depth==0.and.this%right%depth==1) then
          parent => this%right
          parent%right%up=>this
          parent%left%up=>this
          this%left=>parent%left
          this%right=>parent%right
          this%depth=1
          deallocate(parent)
          parent=>this%left
          select type (parent)
          class is (fibonacci_leave_type)
          this%leftmost => parent
          end select
          this%down=>this%leftmost%down
       end if
    end if
    nullify(leave%right%left)
    nullify(leave%up)
    nullify(leave%right)
    nullify(this%leftmost%left)
    !call this%write_pstricks(11)!PSTRICKS
    !flush(11)!PSTRICKS
  end subroutine fibonacci_root_pop_left
\end{Verbatim}

\TbpImp{fibonacci\_root\_pop\_right}
\begin{Verbatim}
  subroutine fibonacci_root_pop_right(this,leave)
    class(fibonacci_root_type),intent(inout),target  :: this
    class(fibonacci_leave_type),pointer,intent(out) :: leave
    class(fibonacci_node_type),pointer  :: parent,grand
    leave => this%rightmost
    if (this%right%depth>=1) then
       parent => leave%up
       grand=>parent%up
       grand%right => parent%left
       parent%left%up=>grand
       deallocate(parent)
       parent=>grand%right
       if (.not.parent%is_leave())then
          parent=>parent%right
       end if
       select type (parent)
       class is (fibonacci_leave_type)
          this%rightmost => parent
       class default
          print *,"fibonacci_root_pop_left: ERROR: leftmost is no leave."
          call parent%print_all()
          STOP
       end select
       call grand%repair()
    else
       if (this%right%depth==0.and.this%left%depth==1) then
          parent => this%left
          parent%left%up=>this
          parent%right%up=>this
          this%right=>parent%right
          this%left=>parent%left
          this%depth=1
          deallocate(parent)
          parent=>this%right
          select type (parent)
          class is (fibonacci_leave_type)
          this%rightmost => parent
          end select
          this%down=>this%rightmost%down
       end if
    end if
  end subroutine fibonacci_root_pop_right
\end{Verbatim}

\TbpImp{fibonacci\_root\_merge}
\begin{Verbatim}
  subroutine fibonacci_root_merge(this_tree,that_tree,merge_tree)
    ! I neither used nor revised this procedure for a long time, so it might be broken.
    class(fibonacci_root_type),intent(in) :: this_tree
    class(fibonacci_root_type),intent(in) :: that_tree
    class(fibonacci_root_type),pointer,intent(out) :: merge_tree
    class(fibonacci_leave_type),pointer :: this_leave,that_leave,old_leave
    type(fibonacci_leave_list_type),target :: leave_list
    class(fibonacci_leave_list_type),pointer :: last_leave
    integer :: n_leaves
    if (associated(this_tree%leftmost).and.associated(that_tree%leftmost)) then
       n_leaves=1
       this_leave=>this_tree%leftmost
       that_leave=>that_tree%leftmost
       if (this_leave < that_leave) then
          allocate(leave_list%leave,source=this_leave)
          call this_leave%find_right_leave(this_leave)
       else
          allocate(leave_list%leave,source=that_leave)
          call that_leave%find_right_leave(that_leave)
       end if
       last_leave=>leave_list
       do while (associated(this_leave).and.associated(that_leave))
          if (this_leave < that_leave) then
             old_leave=>this_leave
             call this_leave%find_right_leave(this_leave)
          else
             old_leave=>that_leave
             call that_leave%find_right_leave(that_leave)
          end if
          allocate(last_leave%next)
          last_leave=>last_leave%next
          allocate(last_leave%leave,source=old_leave)
          n_leaves=n_leaves+1
       end do
       if (associated(this_leave)) then
          old_leave=>this_leave
       else
          old_leave=>that_leave
       end if
       do while (associated(old_leave))
          allocate(last_leave%next)
          last_leave=>last_leave%next
          allocate(last_leave%leave,source=old_leave)
          n_leaves=n_leaves+1
          call old_leave%find_right_leave(old_leave)
       end do
       allocate(merge_tree)
       call fibonacci_root_list_to_tree(merge_tree,n_leaves,leave_list)
    else
       n_leaves=0
    end if
    if(associated(leave_list%next)) then
       last_leave=>leave_list%next
       do while (associated(last_leave%next))
          leave_list%next=>last_leave%next
          deallocate(last_leave)
          last_leave=>leave_list%next
       end do
       deallocate(last_leave)
    end if
  end subroutine fibonacci_root_merge
\end{Verbatim}

\TbpImp{fibonacci\_root\_set\_leftmost}
\begin{Verbatim}
  subroutine fibonacci_root_set_leftmost(this)
    class(fibonacci_root_type) :: this
    call this%find_leftmost(this%leftmost)
  end subroutine fibonacci_root_set_leftmost
\end{Verbatim}

\TbpImp{fibonacci\_root\_set\_rightmost}
\begin{Verbatim}
  subroutine fibonacci_root_set_rightmost(this)
    class(fibonacci_root_type) :: this
    call this%find_rightmost(this%rightmost)
  end subroutine fibonacci_root_set_rightmost
\end{Verbatim}

\TbpImp{fibonacci\_root\_init\_by\_leave}
\begin{Verbatim}
  subroutine fibonacci_root_init_by_leave(this,left_leave,right_leave)
    class(fibonacci_root_type),target,intent(out) :: this
    class(fibonacci_leave_type),target,intent(in) :: left_leave,right_leave
    if (left_leave <= right_leave) then
       this%left  =>  left_leave
       this%right => right_leave
       this%leftmost => left_leave
       this%rightmost => right_leave
    else
       this%left  => right_leave
       this%right  => left_leave
       this%leftmost => right_leave
       this%rightmost => left_leave
    end if
    this%left%up => this
    this%right%up => this
    this%down=>this%leftmost%down
    this%depth = 1
    this%leftmost%right=>this%rightmost
    this%rightmost%left=>this%leftmost
    this%is_valid_c=.true.
  end subroutine fibonacci_root_init_by_leave
\end{Verbatim}

\TbpImp{fibonacci\_root\_init\_by\_content}
\begin{Verbatim}
  subroutine fibonacci_root_init_by_content(this,left_content,right_content)
    class(fibonacci_root_type),target,intent(out) :: this
    class(measurable_class),intent(in),target :: left_content,right_content
    call fibonacci_root_reset(this)
    print *,"fibonacci_root_init_by_content: ",left_content%measure(),right_content%measure()
    if (left_content<right_content) then
       call this%leftmost%set_content(left_content)
       call this%rightmost%set_content(right_content)
    else
       call this%leftmost%set_content(right_content)
       call this%rightmost%set_content(left_content)
    end if
    this%down=>this%leftmost%down
    this%is_valid_c=.true.
  end subroutine fibonacci_root_init_by_content
\end{Verbatim}

\TbpImp{fibonacci\_root\_reset}
\begin{Verbatim}
  subroutine fibonacci_root_reset(this)
    class(fibonacci_root_type),target,intent(inout) :: this
    call fibonacci_root_deallocate_tree(this)
    allocate (this%leftmost)
    allocate (this%rightmost) 
    this%depth=1
    this%leftmost%depth=0
    this%rightmost%depth=0
    this%left=>this%leftmost
    this%right=>this%rightmost
    this%left%up=>this
    this%right%up=>this
    this%leftmost%right=>this%rightmost
    this%rightmost%left=>this%leftmost
  end subroutine fibonacci_root_reset
\end{Verbatim}

\TbpImp{fibonacci\_root\_deallocate\_tree}
\begin{Verbatim}
  recursive subroutine fibonacci_root_deallocate_tree(this)
    class(fibonacci_root_type),intent(inout) :: this
    call fibonacci_node_deallocate_tree(this)
    nullify(this%leftmost)
    nullify(this%rightmost)
  end subroutine fibonacci_root_deallocate_tree
\end{Verbatim}

\TbpImp{fibonacci\_root\_deallocate\_all}
\begin{Verbatim}
  recursive subroutine fibonacci_root_deallocate_all(this)
    class(fibonacci_root_type),intent(inout) :: this
    call fibonacci_node_deallocate_all(this)
    nullify(this%leftmost)
    nullify(this%rightmost)
  end subroutine fibonacci_root_deallocate_all
\end{Verbatim}

\TbpImp{fibonacci\_root\_is\_left\_child}
\begin{Verbatim}
  elemental logical function fibonacci_root_is_left_child(this)
    class(fibonacci_root_type),target,intent(in) :: this
    fibonacci_root_is_left_child = .false.
  end function fibonacci_root_is_left_child
\end{Verbatim}

\TbpImp{fibonacci\_root\_is\_right\_child}
\begin{Verbatim}
  elemental logical function fibonacci_root_is_right_child(this)
    class(fibonacci_root_type),target,intent(in) :: this
    fibonacci_root_is_right_child = .false.
  end function fibonacci_root_is_right_child
\end{Verbatim}

\MethodsFor{fibonacci\_stub\_type}
\TbpImp{fibonacci\_stub\_get\_type}
\begin{Verbatim}
  pure subroutine fibonacci_stub_get_type(type)
    character(:),allocatable,intent(out)::type
    allocate(type,source="fibonacci_stub_type")
  end subroutine fibonacci_stub_get_type
\end{Verbatim}

\TbpImp{fibonacci\_stub\_push\_by\_content}
\begin{Verbatim}
  subroutine fibonacci_stub_push_by_content(this,content)
    class(fibonacci_stub_type),target,intent(inout)  :: this
    class(measurable_class),target,intent(in)::content
    class(fibonacci_leave_type),pointer::leave
    allocate(leave)
    call leave%set_content(content)
    call this%push_by_leave(leave)
  end subroutine fibonacci_stub_push_by_content
\end{Verbatim}
  
\TbpImp{fibonacci\_stub\_push\_by\_leave}
\begin{Verbatim}
  subroutine fibonacci_stub_push_by_leave(this,new_leave)
    class(fibonacci_stub_type),target,intent(inout)   :: this
    class(fibonacci_leave_type),pointer,intent(inout) :: new_leave
    class(fibonacci_leave_type),pointer::old_leave
    if(this%depth<1)then
       if(associated(this%leftmost))then     
          old_leave=>this%leftmost
          call this%init_by_leave(old_leave,new_leave)
       else
          this%leftmost=>new_leave
       end if
    else
       call fibonacci_root_push_by_leave(this,new_leave)
    end if
  end subroutine fibonacci_stub_push_by_leave
\end{Verbatim}

\TbpImp{fibonacci\_stub\_pop\_left}
\begin{Verbatim}
  subroutine fibonacci_stub_pop_left(this,leave)
    class(fibonacci_stub_type),intent(inout),target  :: this
    class(fibonacci_leave_type),pointer,intent(out) :: leave
    if(this%depth<2)then
       if(this%depth==1)then
          leave=>this%leftmost
          this%leftmost=>this%rightmost
          nullify(this%rightmost)
          nullify(this%right)
          nullify(this%left)
          this%depth=0
          this%is_valid_c=.false.
       else
          if(associated(this%leftmost))then
             leave=>this%leftmost
             nullify(this%leftmost)
          end if
       end if
    else
       call fibonacci_root_pop_left(this,leave)
    end if
  end subroutine fibonacci_stub_pop_left
\end{Verbatim}

\TbpImp{fibonacci\_stub\_pop\_right}
\begin{Verbatim}
  subroutine fibonacci_stub_pop_right(this,leave)
    class(fibonacci_stub_type),intent(inout),target  :: this
    class(fibonacci_leave_type),pointer,intent(out) :: leave
    if(this%depth<2)then
       if(this%depth==1)then
          this%is_valid_c=.false.
          if(associated(this%rightmost))then
             leave=>this%rightmost
             nullify(this%rightmost)
             nullify(this%right)
          else
             if(associated(this%leftmost))then
                leave=>this%leftmost
                nullify(this%leftmost)
                nullify(this%left)
             else
                nullify(leave)
             end if
          end if
       end if
    else
       call fibonacci_root_pop_right(this,leave)
    end if
  end subroutine fibonacci_stub_pop_right
\end{Verbatim}

\MethodsFor{fibonacci\_leave\_type}

\TbpImp{fibonacci\_leave\_get\_type}
\begin{Verbatim}
  pure subroutine fibonacci_leave_get_type(type)
    character(:),allocatable,intent(out)::type
    allocate(type,source="fibonacci_leave_type")
  end subroutine fibonacci_leave_get_type
\end{Verbatim}

\TbpImp{fibonacci\_leave\_print\_to\_unit}
\begin{Verbatim}
  subroutine fibonacci_leave_print_to_unit(this,unit,parents,components,peers)
    class(fibonacci_leave_type),intent(in)::this
    integer,intent(in)::unit
    integer(kind=dik),intent(in)::parents,components,peers
    class(serializable_class),pointer::ser
    if(parents>0)call fibonacci_node_print_to_unit(this,unit,parents-one,components,-one)
    write(unit,'("Components of fibonacci_leave_type:")')
    ser=>this%down
    call serialize_print_comp_pointer(ser,unit,parents,components,peers,"Content:")
  end subroutine fibonacci_leave_print_to_unit
\end{Verbatim}

\TbpImp{fibonacci\_leave\_get\_left}
\begin{Verbatim}
  subroutine fibonacci_leave_get_left(this,leave)
    class(fibonacci_leave_type),intent(in) :: this
    class(fibonacci_leave_type),intent(out),pointer :: leave
    class(fibonacci_node_type),pointer::node
    node=>this%left
    select type(node)
    class is (fibonacci_leave_type)
       leave=>node
    end select
  end subroutine fibonacci_leave_get_left
\end{Verbatim}
  
\TbpImp{fibonacci\_leave\_get\_right}
\begin{Verbatim}
  subroutine fibonacci_leave_get_right(this,leave)
    class(fibonacci_leave_type),intent(in) :: this
    class(fibonacci_leave_type),intent(out),pointer :: leave
    class(fibonacci_node_type),pointer::node
!    print *,"fibonacci_leave_get_right"
!    call this%down%print_little
    if(associated(this%right))then
       node=>this%right
!       call node%down%print_little
       select type(node)
       class is (fibonacci_leave_type)
          leave=>node
       end select
    else
!       print *,"no right leave"
       nullify(leave)
    end if
  end subroutine fibonacci_leave_get_right
\end{Verbatim}
  
\TbpImp{fibonacci\_leave\_deallocate\_all}
\begin{Verbatim}
  subroutine fibonacci_leave_deallocate_all(this)
    class(fibonacci_leave_type),intent(inout) :: this
    if (associated(this%down)) then
       deallocate(this%down)
    end if
  end subroutine fibonacci_leave_deallocate_all
\end{Verbatim}

\TbpImp{fibonacci\_leave\_write\_pstricks}
\begin{Verbatim}
  subroutine fibonacci_leave_write_pstricks(this,unitnr)
    class(fibonacci_leave_type),intent(in),target :: this
    integer,intent(in) :: unitnr
    write(unitnr,'("{\textbackslash}begin{psTree}{{\textbackslash}Toval[linecolor=green]{{\textbackslash}node{",i3,"}{",f9.3,"}}}")')&
      this%depth,this%measure()
    if (associated(this%left)) then
       write(unitnr,'("{\textbackslash}Tr[",a,"]{}")') le_kid
    end if
    if (associated(this%right)) then
       write(unitnr,'("{\textbackslash}Tr[",a,"]{}")') le_kid
    end if
    write(unitnr,'("{\textbackslash}end{psTree}")')
  end subroutine fibonacci_leave_write_pstricks
\end{Verbatim}

\TbpImp{fibonacci\_leave\_insert\_leave\_by\_node}
\begin{Verbatim}
  subroutine fibonacci_leave_insert_leave_by_node(this,new_leave)
    class(fibonacci_leave_type),target,intent(inout) :: this,new_leave
    class(fibonacci_node_type),pointer :: parent,new_node
    parent=>this%up
    !print *,associated(this%left),associated(this%right)
    if(this<new_leave)then
       call fibonacci_node_spawn(new_node,this,new_leave,this%left,this%right)
       !print *,"Repair! ",this%measure(),new_leave%measure()
    else
       call fibonacci_node_spawn(new_node,new_leave,this,this%left,this%right)
    end if
    if(associated(parent%left,this))then
       call parent%append_left(new_node)
    else
       call parent%append_right(new_node)
    end if
    call parent%repair()
  end subroutine fibonacci_leave_insert_leave_by_node
\end{Verbatim}

\TbpImp{fibonacci\_leave\_copy\_content}
\begin{Verbatim}
  subroutine fibonacci_leave_copy_content(this,content)
    class(fibonacci_leave_type) :: this
    class(measurable_class),intent(in) :: content
    allocate(this%down,source=content)
  end subroutine fibonacci_leave_copy_content
\end{Verbatim}

\TbpImp{fibonacci\_leave\_set\_content}
\begin{Verbatim}
  subroutine fibonacci_leave_set_content(this,content)
    class(fibonacci_leave_type) :: this
    class(measurable_class),target,intent(in) :: content
    this%down => content
  end subroutine fibonacci_leave_set_content
\end{Verbatim}

\TbpImp{fibonacci\_leave\_get\_content}
\begin{Verbatim}
  subroutine fibonacci_leave_get_content(this,content)
    class(fibonacci_leave_type),intent(in) :: this
    class(measurable_class),pointer :: content
    content => this%down
  end subroutine fibonacci_leave_get_content
\end{Verbatim}

\TbpImp{fibonacci\_leave\_is\_inner}
\begin{Verbatim}
  elemental logical function fibonacci_leave_is_inner()
    fibonacci_leave_is_inner = .false.
  end function fibonacci_leave_is_inner
\end{Verbatim}

\TbpImp{fibonacci\_leave\_is\_leave}
\begin{Verbatim}
  elemental logical function fibonacci_leave_is_leave()
    fibonacci_leave_is_leave = .true.
  end function fibonacci_leave_is_leave
\end{Verbatim}

\TbpImp{fibonacci\_leave\_is\_left\_short}
\begin{Verbatim}
  elemental logical function fibonacci_leave_is_left_short(this)
    class(fibonacci_leave_type),intent(in) :: this
    fibonacci_leave_is_left_short = .false.
  end function fibonacci_leave_is_left_short
\end{Verbatim}

\TbpImp{fibonacci\_leave\_is\_right\_short}
\begin{Verbatim}
  elemental logical function fibonacci_leave_is_right_short(this)
    class(fibonacci_leave_type),intent(in) :: this
    fibonacci_leave_is_right_short = .false.
  end function fibonacci_leave_is_right_short
\end{Verbatim}

\TbpImp{fibonacci\_leave\_is\_unbalanced}
\begin{Verbatim}
  elemental logical function fibonacci_leave_is_unbalanced(this)
    class(fibonacci_leave_type),intent(in) :: this
    fibonacci_leave_is_unbalanced = .false.
  end function fibonacci_leave_is_unbalanced
\end{Verbatim}

\TbpImp{fibonacci\_leave\_is\_left\_too\_short}
\begin{Verbatim}
  elemental logical function fibonacci_leave_is_left_too_short(this)
    class(fibonacci_leave_type),intent(in) :: this
    fibonacci_leave_is_left_too_short = .false.
  end function fibonacci_leave_is_left_too_short
\end{Verbatim}

\TbpImp{fibonacci\_leave\_is\_right\_too\_short}
\begin{Verbatim}
  elemental logical function fibonacci_leave_is_right_too_short(this)
    class(fibonacci_leave_type),intent(in) :: this
    fibonacci_leave_is_right_too_short = .false.
  end function fibonacci_leave_is_right_too_short
\end{Verbatim}

\TbpImp{fibonacci\_leave\_is\_too\_unbalanced}
\begin{Verbatim}
  elemental logical function fibonacci_leave_is_too_unbalanced(this)
    class(fibonacci_leave_type),intent(in) :: this
    fibonacci_leave_is_too_unbalanced = .false.
  end function fibonacci_leave_is_too_unbalanced
\end{Verbatim}
\MethodsNTB

\ProcImp{fibonacci\_leave\_write\_content}
\begin{Verbatim}
  subroutine fibonacci_leave_write_content(this,unit)
    class(fibonacci_leave_type),intent(in),target :: this
    integer,optional,intent(in)::unit
    call this%down%print_all(unit)
  end subroutine fibonacci_leave_write_content
\end{Verbatim}

\ProcImp{fibonacci\_leave\_write}
\begin{Verbatim}
  subroutine fibonacci_leave_write(this,unit)
    class(fibonacci_leave_type),intent(in),target :: this
    integer,optional,intent(in)::unit
    call this%print_all(unit)
  end subroutine fibonacci_leave_write
\end{Verbatim}

\ProcImp{fibonacci\_leave\_write\_value}
\begin{Verbatim}
  subroutine fibonacci_leave_write_value(this,unit)
    class(fibonacci_leave_type),intent(in),target :: this
    integer,intent(in),optional::unit
    if(present(unit))then
       write(unit,fmt=*)this%measure()
    else
       print *,this%measure()
    end if
!    call this%print_little(unit)
  end subroutine fibonacci_leave_write_value
\end{Verbatim}

\ProcImp{fibonacci\_node\_spawn}
\begin{Verbatim}
  subroutine fibonacci_node_spawn&
    (new_node,left_leave,right_leave,left_left_leave,right_right_leave)
    class(fibonacci_node_type),pointer,intent(out) :: new_node
    class(fibonacci_leave_type),target,intent(inout) :: left_leave,right_leave
    class(fibonacci_node_type),pointer,intent(inout) :: left_left_leave,right_right_leave
    allocate(new_node)
    new_node%depth=1
    if(associated(left_left_leave))then
       left_left_leave%right=>left_leave
       left_leave%left=>left_left_leave
    else
       nullify(left_leave%left)
    end if
    if(associated(right_right_leave))then
       right_right_leave%left=>right_leave
       right_leave%right=>right_right_leave
    else
       nullify(right_leave%right)
    end if
    new_node%left=>left_leave
    new_node%right=>right_leave
    new_node%down=>left_leave%down
    new_node%depth=1
    left_leave%up=>new_node
    right_leave%up=>new_node
    left_leave%right=>right_leave
    right_leave%left=>left_leave
  end subroutine fibonacci_node_spawn
\end{Verbatim}

\ProcImp{fibonacci\_root\_list\_to\_tree}
\begin{Verbatim}
  subroutine fibonacci_root_list_to_tree(this,n_leaves,leave_list_target)
    class(fibonacci_root_type),target :: this
    integer,intent(in) :: n_leaves
    type(fibonacci_leave_list_type),target,intent(in) :: leave_list_target
!    class(fibonacci_root_type),pointer,intent(out) :: tree
    integer:: depth,n_deep,n_merge
    class(fibonacci_node_type),pointer :: node
    class(fibonacci_leave_list_type),pointer :: leave_list
    class(fibonacci_leave_type),pointer::content
    real(kind=double) :: up_value
    leave_list=>leave_list_target
    call ilog2(n_leaves,depth,n_deep)
    n_deep=n_deep*2
    n_merge=0
    this%depth=depth
    node=>this
    outer: do
       do while(depth>1)
          depth=depth-1
          allocate(node%left)
          node%left%up=>node
          node=>node%left
          node%depth=depth
       end do
       node%left=>leave_list%leave
       node%down=>leave_list%leave%down
       leave_list=>leave_list%next
       node%right=>leave_list%leave
       content => leave_list%leave
       leave_list=>leave_list%next
       n_merge=n_merge+2
       inner: do
          if (associated(node%up)) then
             if (node%is_left_child()) then
                if (n_merge==n_deep.and.depth==1) then
                   node=>node%up
                   node%right=>leave_list%leave
                   node%right%up=>node
                   node%down=>content%down
                   content=>leave_list%leave
                   leave_list=>leave_list%next
                   n_merge=n_merge+1
                   cycle
                end if
                exit inner
             else
                node=>node%up
                depth=depth+1
             end if
          else
             exit outer
          end if
       end do inner
       node=>node%up
       node%down=>content%down
       allocate(node%right)
       node%right%up => node
       node=>node%right
       if (n_deep==n_merge) then
          depth=depth-1
       end if
       node%depth=depth
    end do outer
    call this%set_leftmost
    call this%set_rightmost
  end subroutine fibonacci_root_list_to_tree
\end{Verbatim}

