%%%%%%%%%%%%%%%%%%%%%%%%%%%%%%%%%%%%%%%%%%%%%%%%%%%%%%%%%%%%%%%%%%%%%%%%
% Manual for:
% gamelan:  Graphical Analysis macros for the Metapost Language
%%%%%%%%%%%%%%%%%%%%%%%%%%%%%%%%%%%%%%%%%%%%%%%%%%%%%%%%%%%%%%%%%%%%%%%%
%\NeedsTeXFormat{LaTeX2e}
\documentclass[12pt,a4paper]{report}
\usepackage{doc,color,gamelan}
%%%%%%%%%%%%%%%%%%%%%%%%%%%%%%%%%%%%%%%%%%%%%%%%%%%%%%%%%%%%%%%%%%%%%%%%
%%% Package specific information
%\listfiles
\def\examples#1{%
  \begin{list}{}%
   {\setlength{\leftmargin}{#1}%
    \addtolength{\leftmargin}{\marginparsep}%
    \addtolength{\leftmargin}{-\marginparwidth}%
    \setlength{\rightmargin}{0mm}%
    \setlength{\itemindent}{\parindent}%
    \setlength{\listparindent}{\parindent}}%
    \item}
\def\endexamples{\end{list}}
\def\marginexample#1{\marginpar{\hbox to\marginparwidth{#1\hss}}}
%\parindent0pt
\def\manindex#1{\SortIndex{#1}{#1}}
\OnlyDescription
\EnableCrossrefs
\RecordChanges
\CodelineIndex
%\DoNotIndex{\def,\gdef,\long,\let,\begin,\end,\if,\ifx,\else,\fi}
%\DoNotIndex{\immediate,\write,\newwrite,\openout,\closeout,\typeout}
%\DoNotIndex{\font,\nullfont,\jobname,\documentclass}
%\DoNotIndex{\batchmode,\errorstopmode,\char,\catcode,\ }
%\DoNotIndex{\CodelineIndex,\DocInput,\DoNotIndex,\EnableCrossrefs}
%\DoNotIndex{\filedate,\filename,\fileversion,\logo,\manfnt}
%\DoNotIndex{\NeedsTeXFormat,\ProvidesPackage,\RecordChanges,\space}
%\DoNotIndex{\usepackage,\wlog,\@gobble,\@ifundefined,\@namedef,\@spaces}
%\DoNotIndex{\begingroup,\csname,\edef,\endcsname,\expandafter,\hbox}
%\DoNotIndex{\hskip,\ifeof,\ignorespaces,\item,\leavevmode,\loop,\makebox}
%\DoNotIndex{\newcounter,\newif,\newread,\openin,\par,\parindent,\put}
%\DoNotIndex{\read,\relax,\repeat,\setcounter,\stepcounter,\the}
%\DoNotIndex{\value,\vbox,\vskip}
%\DoNotIndex{\@tempa,\@tempb,\advance,\bf,\bfseries,\closein}
%\DoNotIndex{\CurrentOption,\DeclareOption,\documentstyle,\em,\emph}
%\DoNotIndex{\endgroup,\epsilon,\global,\hfuzz,\input,\it,\LaTeX,\LaTeXe}
%\DoNotIndex{\macrocode,\meaning,\OnlyDescription,\PassOptionsToPackage}
%\DoNotIndex{\ProcessOptions,\RequirePackage,\sf,\string,\textbf,\textit}
%\DoNotIndex{\textsf,\texttt,\tt,\unitlength}
\let\origmacrocode\macrocode
\def\macrocode{\hfuzz 5em\origmacrocode}

%%%%%%%%%%%%%%%%%%%%%%%%%%%%%%%%%%%%%%%%%%%%%%%%%%%%%%%%%%%%%%%%%%%%%%%%
%%% Macro section
%%%%%%%%%%%%%%%%%%%%%%%%%%%%%%%%%%%%%%%%%%%%%%%%%%%%%%%%%%%%%%%%%%%%%%%%
\makeatletter
\newif\if@preliminary
\@preliminaryfalse
\def\preliminary{\@preliminarytrue}
%%%%%%%%%%%%%%%%%%%%%%%%%%%%%%%%%%%%%%%%%%%%%%%%%%%%%%%%%%%%%%%%%%%%%%%%
%%% Changes referring to article.cls
%
%%% Title page
\def\preprintno#1{\def\@preprintno{#1}}
\def\address#1{\def\@address{#1}}
\def\email#1{\thanks{\tt #1}}
\def\abstract#1{\def\@abstract{#1}}
\renewcommand\abstractname{ABSTRACT}
\newlength\preprintnoskip
\setlength\preprintnoskip{\textwidth\@plus -1cm}
\newlength\abstractwidth
\setlength\abstractwidth{\textwidth\@plus -3cm}
%
\@titlepagetrue
\renewcommand\maketitle{\begin{titlepage}%
  \let\footnotesize\small
  \hfill\parbox{\preprintnoskip}{%
  \begin{flushright}\@preprintno\end{flushright}}\hspace*{1cm}
  \vskip 60\p@
  \begin{center}%
    {\Large\bf\boldmath \@title \par}\vskip 1cm%
    {\sc\@author \par}\vskip 3mm%
    {\@address \par}%
    \if@preliminary
      \vskip 2cm {\large\sf PRELIMINARY DRAFT \par \@date}%
    \fi
  \end{center}\par
  \@thanks
  \vfill
  \begin{center}%
    \parbox{\abstractwidth}{\centerline{\abstractname}%
    \vskip 3mm%
    \@abstract}
  \end{center}
  \end{titlepage}%
  \setcounter{footnote}{0}%
  \let\thanks\relax\let\maketitle\relax
  \gdef\@thanks{}\gdef\@author{}\gdef\@address{}%
  \gdef\@title{}\gdef\@abstract{}\gdef\@preprintno{}
}%
%
%%% New settings of dimensions
\topmargin -1.5cm
\textheight 22cm
\textwidth 13cm
\oddsidemargin 2cm
\evensidemargin 2cm
%
%%% Captions set in italics
\long\def\@makecaption#1#2{%
  \vskip\abovecaptionskip
  \sbox\@tempboxa{#1: \emph{#2}}%
  \ifdim \wd\@tempboxa >\hsize
    #1: \emph{#2}\par
  \else
    \hbox to\hsize{\hfil\box\@tempboxa\hfil}%
  \fi
  \vskip\belowcaptionskip}
%
\makeatother

%%%%%%%%%%%%%%%%%%%%%%%%%%%%%%%%%%%%%%%%%%%%%%%%%%%%%%%%%%%%%%%%%%%%%%%%
\begin{gmlpreamble}
\usepackage{amsfonts}
\end{gmlpreamble}
%%%%%%%%%%%%%%%%%%%%%%%%%%%%%%%%%%%%%%%%%%%%%%%%%%%%%%%%%%%%%%%%%%%%%%%%
\def\Red#1{\textcolor{red}{#1}}
\def\Green#1{\textcolor{green}{#1}}
\def\Blue#1{\textcolor{blue}{#1}}
\def\Cyan#1{\textcolor{cyan}{#1}}
\def\Magenta#1{\textcolor{magenta}{#1}}
\def\Yellow#1{\textcolor{yellow}{#1}}
\def\Black#1{\textcolor{black}{#1}}
\def\White#1{\textcolor{white}{#1}}
\definecolor{chartreuse}{rgb}{0.5, 1, 0} 
%%%%%%%%%%%%%%%%%%%%%%%%%%%%%%%%%%%%%%%%%%%%%%%%%%%%%%%%%%%%%%%%%%%%%%%%
%%% Abbrevs
\def\MF{\textsf{META}\-\textsf{FONT}}%
\def\MP{\textsf{Meta}\-\textsf{Post}}%
\def\GML{\textsf{gamelan}}%
\def\FMF{\texttt{feyn}\textsf{MF}}%
\def\PS{\texttt{PostScript}}
%%%%%%%%%%%%%%%%%%%%%%%%%%%%%%%%%%%%%%%%%%%%%%%%%%%%%%%%%%%%%%%%%%%%%%%%
%%% Titlepage
%%%%%%%%%%%%%%%%%%%%%%%%%%%%%%%%%%%%%%%%%%%%%%%%%%%%%%%%%%%%%%%%%%%%%%%%
\begin{document}
\begin{gmlfile}
\preliminary        % mark on title page
%\baselineskip20pt   % stretch linespacing in main text
%%%%%%%%%%%%%%%%%%%%%%%%%%%%%%%%%%%%%%%%%%%%%%%%%%%%%%%%%%%%%%%%%%%%%%%%
\preprintno{November 1997}
\title{%
 \GML:\\[.5\baselineskip]
 Graphical Analysis macros\\ for the \MP\ Language\\[.5\baselineskip]
 MANUAL
}
\author{%  
 Wolfgang Kilian%
 \email{kilian@physik.uni-siegen.de}
}
\address{%
 Theoretische Physik 1\\
 Walter-Flex-Str. 3, Universit\"at Siegen\\
 D-57068 Siegen, Germany
}
\abstract{%
\GML\ is a package for data and function plotting within a \LaTeX\
document.  It is based on macros written in \MP, originally derived
from John Hobby's `graph.mp` macro package.
This manual describes version 0.40.
}
%
\maketitle
\tableofcontents
%%%%%%%%%%%%%%%%%%%%%%%%%%%%%%%%%%%%%%%%%%%%%%%%%%%%%%%%%%%%%%%%%%%%%%
%%% Text
%%%%%%%%%%%%%%%%%%%%%%%%%%%%%%%%%%%%%%%%%%%%%%%%%%%%%%%%%%%%%%%%%%%%%%%%
\MakeShortVerb{`}
%
\chapter{The \LaTeX\ interface}
\section{Using \GML}
\GML\ consists of two parts: A \LaTeX\ style file and a \MP\ macro
package.  The style file implements two environments at top level:
`gmlpreamble` and `gmlfile`.  The former is part of the \LaTeX\
preamble; it is used to pass certain macro
declarations down to \GML.  The `gmlfile` environment is part of the
main document.  It encloses the actual graphs to be drawn by \GML.

Inside a `gmlfile` environment, four additional environments are
defined: `gmlcode`, `gmlfigure`, `gmlgraph`, and `gmltex`.  The text
enclosed by these is not parsed by the \LaTeX\ interpreter, but written
verbatim to a file.  This file serves as an input file to \GML, which
transforms it into `eps` (encapsulated postscript) figures.  The
figures are automatically inserted into the original document
in an additional \LaTeX\ run.

The document structure is depicted schematically in
Fig.\ref{fig:document}.  There is at most one `gmlpreamble`,
but there may be several `gmlfile` environments, each one producing a
separate input file for \GML.  Inside a `gmlfile`, the number of
graphs and figures is unlimited.

To load the style file, put a corresponding `\usepackage` declaration
in the document header:
\begin{quote}
`\usepackage{gamelan}`
\end{quote}
Since \GML\ uses the standard `verbatim` and `graphics` packages,
those two need not be called separately if you need them; they are
already present.


\section{The \texttt{gmlfile} environment}
\DescribeEnv{gmlfile} The `gmlfile` environment activates \GML's
functionality by defining the figure-drawing environments, and by
specifying where \GML\ commands should be written to.  It has an
optional argument which specifies a filename:
\begin{quote}
`\begin{gmlfile}` \qquad or\\
`\begin{gmlfile}[`\meta{filename}`]`
\end{quote}
If the optional argument is omitted, the \meta{filename} is set to the
filename of the main document, given by `\jobname`.  Any code inside
the graph-drawing environments will be written to the file with the
specified name and extension `.mp`.

The `gmlfile` environment should be placed inside the document's
`\begin{document}` \ldots `\end{document}` body.  Apart from
\GML\ figures and graphs, there can be arbitrary
text, figures, sections, chapters, etc., inside it.  In a
short document where only a single `gmlfile` environment is used, the
`\begin{gmlfile}` command may immediately follow `\begin{document}`.
However, in a long document it is more convenient to use several
`gmlfile`s with different names and to restrict their scope to a
particular section or figure.


\subsection{Writing \GML\ code}
\GML\ interprets code written in \MP\ language.  This language is not
intellegible to \LaTeX.  However, the environments described in this
section provide an interface such that \MP\ code can inserted in a
\LaTeX\ document.  On the other hand, \GML\ will envoke \LaTeX\ to
format textual labels.

\DescribeEnv{gmlcode} The simplest environment is `gmlcode`:
\begin{quote}
`\begin{gmlcode}`\\
`  `\meta{\GML\ declarations and commands}\\
`\end{gmlcode}`
\end{quote}
This environment does not produce any graphical output; it encloses
global declarations and similar things .  For instance, you may wish to
define global variables and macros:
\begin{quote}
`\begin{gmlcode}`\\
`  scale:= 1000;`\\
`  vardef foo(expr x) = x + 42 enddef;`\\
`\end{gmlcode}`
\end{quote}
or a dataset is read in, to be used in several distinct
graphs:
\begin{quote}
`\begin{gmlcode}`\\
`  directory "data/";`\\
`  fromfile "run1.dat":`\\
`    table plot(run1)();`\\
`  endfrom`\\
`\end{gmlcode}`
\end{quote}
The `gmlcode` environment is also a way to try out some of the
examples in this manual which do not involve graphical output:
\begin{quote}
`\begin{gmlcode}`\\
`  a=4; show 3a;`\\
`\end{gmlcode}`
\end{quote}
The `gmlcode` environment must reside inside a `gmlfile`, so the
latter example reads:
\begin{quote}
`\documentclass{article}`\\
`\usepackage{gamelan}`\\
`\begin{document}`\\
`\begin{gmlfile}`\\
`\begin{gmlcode}`\\
`  a=4; show 3a;`\\
`\end{gmlcode}`\\
`\end{gmlfile}`\\
`\end{document}`
\end{quote}
If you type this code in your favorite editor, save it into a file
named, e.g., `foo.tex`, and start \LaTeX\
\begin{quote}
`> latex foo`
\end{quote}
a file `foo.mp` is created containing, among other stuff, the line
enclosed between `\begin{gmlcode}` and `\end{gmlcode}`.  Now write
\begin{quote}
`> gml foo`
\end{quote}
and \GML\ is started with the input file `foo.mp`, resulting in the
output
\begin{quote}
\begin{verbatim}
This is MetaPost, Version 0.631 (C version 6.1)
(foo.mp
Package: `gamelan' 0.40 <1998/11/02>
>> 12 )
Transcript written on foo.log.
\end{verbatim}
\end{quote}
The number `12` is the result of the `show 3a;` command.  Clearly,
neither graphics nor text has been generated.  We will come to real
figures in a minute.

\subsection{Figures}
\DescribeMacro{gmlfigure} The `gmlfigure` environment is a
wrapper for \GML\ code intended to generate graphical output:
\begin{quote}
`\begin{gmlfigure}`\\
`  `\meta{\GML\ code}\\
`\end{gmlfigure}`
\end{quote}
The \GML\ code is written to file, to be executed by the interpreter,
and the result transformed into an encapsulated Postscript file.  Here
is an example:
\begin{quote}
`\documentclass{article}`\\
`\usepackage{gamelan}`\\
`\begin{document}`\\
`\begin{gmlfile}`\\
`\begin{gmlfigure}`\\
`  draw pentagram scaled 2cm;`\\
`\end{gmlfigure}`\\
`\end{gmlfile}`\\
`\end{document}`
\end{quote}
This file (named, again, `foo.tex`) should be run through \LaTeX
\begin{quote}
`> latex foo`
\end{quote}
\LaTeX\ will create a file `foo.mp`, and it will complain that the figure
`foo.1` has not been found.  To generate this figure, run \GML\ on `foo.mp`
\begin{quote}
`> gml foo`
\end{quote}
with the screen output
\begin{quote}
\begin{verbatim}
This is MetaPost, Version 0.631 (C version 6.1)
(foo.mp
Package: `gamelan' 0.40 <1998/11/02>
[1] )
1 output file written: foo.1
Transcript written on foo.log.
\end{verbatim}
\end{quote}
Now, in a second \LaTeX\ run the figure `foo.1` is included
in the document:
\begin{quote}
`> latex foo`
\end{quote}
and the result shows the desired pentagram
\begin{quote}
`> xdvi foo`\\
``\qquad\qquad\qquad 
\begin{gmlfigure}
  draw pentagram scaled 2cm;
\end{gmlfigure}
\end{quote}

\subsection{Graphs}
\DescribeEnv{gmlgraph} Now that we are able to produce graphics, why
is there a second environment for the same purpose?  The reason is, that
for the task of visualizing data, you need to specify the dimensions of
the graph in advance: \GML\ knows the size of `1cm` on paper, but it
has to be told how to translate a temperature value of $35^\circ$F.

Thus, the `gmlgraph` environment has the same syntax as the familiar
`picture`: The dimensions of the graph on paper are specified in round
brackets, in terms of `\unitlength`.  By default, this is equal to
`1pt`; it is usually a good idea to set it explicitly immediately
before the graph environment\footnote{Usually, one
wraps additional \texttt{figure} and \texttt{center} environments,
or the like, around a graph.  In that case,
\texttt{$\backslash$unitlength} may be reset and you need the explicit
declaration if you do not like to think in \texttt{pt}.}
\begin{quote}
`\documentclass{article}`\\
`\usepackage{gamelan}`\\
`\begin{document}`\\
`\begin{gmlfile}`\\
`\unitlength 1mm`\\
`\begin{gmlgraph}(60,30)`\\
`  draw plot((#0,#1), (#1,#2), (#2,#6),`\\
`            (#3,#1), (#4,#2), (#5,#1));`\\
`\end{gmlgraph}`\\
`\end{gmlfile}`\\
`\end{document}`
\end{quote}
If you run this example through \LaTeX\ and \GML
\begin{quote}
`> latex foo; gml foo; latex foo`
\end{quote}
and try to view the result
\begin{quote}
`> xdvi foo`
\end{quote}
the \PS\ interpreter inside `xdvi` will probably crash.  Similarly,
you will not be able to view or print the EPS file `foo.1` by itself.
The reason is missing font information: In order keep the size of the
EPS files small, \MP\ tells the \PS\ interpreter to take its font
information from the main document, which does not yet exist in \PS\
format.  This is easily remedied by transforming the main document
into \PS
\begin{quote}
`> dvips foo`
\end{quote}
and to display the result as usual
\begin{quote}
`> ghostview foo`\\
``\qquad\qquad\qquad
\unitlength 1mm
\begin{gmlgraph}(60,30)
  draw plot((#0,#1), (#1,#2), (#2,#6), (#3,#1), (#4,#2), (#5,#1));
\end{gmlgraph}
\end{quote}
Now the graph, including axis labels, is visible, and can be printed.

This extra translation step has to be repeated anytime you want to
have a look at the graphics.  However, with this behavior the `.dvi`
file is \emph{really} device-independent and can be distributed, to be
translated into \PS\ on anybody's local machine.  It is guaranteed
that the fonts inside \GML\ graphs match the text fonts in the same
document, and are appropriate for the local printer if the local
`dvips` driver has been set up correctly.  This is not the case for
self-contained EPS files.


\subsection{\LaTeX\ embedded in \GML\ figures}
For typesetting labels, \GML\ calls \LaTeX\ in the background.  This
is normally invisible to the user (except for some delay in processing
the input file).  However, since this background process is distinct
from the \LaTeX\ run which typesets the main document, there must be
some means to communicate macros and style settings to the
subprocess.  This is illustrated in the following example:
\begin{quote}
`\documentclass{article}`\\
`\usepackage{gamelan}`\\
`\begin{gmlpreamble}`\\
`  \usepackage{amsfonts}`\\
`\end{gmlpreamble}`\\
`\begin{document}`\\
`\begin{gmlfile}`\\
`\begin{gmltex}\large\end{gmltex}`\\
`\begin{gmlfigure}`\\
`  draw (-10,0)--(150,0) witharrow;`\\
`  draw (0,-10)--(0,50) witharrow;`\\
`  label(<<$\mathbb{C}$ (the complex plane)>>,`\\
`        (80,40));`\\
`\end{gmlfigure}`\\
`\end{gmlfile}`\\
`\end{document}`
\end{quote}
This has to be processed by \LaTeX\ and \GML, as usual.  \GML\ will
take care of the embedded \LaTeX\ sequence (the text enclosed between
`<<` and `>>`):
\begin{quote}
`> latex foo; gml foo; latex foo; dvips foo`\\
`> ghostview foo`\\[.5\baselineskip]
``\qquad\qquad\qquad
\begin{gmltex}
\large
\end{gmltex}
\begin{gmlfigure}
  draw (-10,0)--(150,0) witharrow;
  draw (0,-10)--(0,50) witharrow;
  label(btex $\mathbb{C}$ (the complex plane) etex, (80,40));
\end{gmlfigure}
\end{quote}
\DescribeEnv{gmlpreamble} Here, \GML\ has to access the letter C in
the `amsfonts` package.  This is achieved by wrapping the
corresponding `\usepackage` command in a `gmlpreamble` environment.
The code enclosed in this environment is written verbatim to an
auxiliary file with extension `.ltp` (`foo.ltp` in this case).  This
file is included at the end of the preamble both by the main document
and by any \LaTeX\ subprocess which is started by `gamelan`.

Several `gmlpreamble` environments may appear in the preamble of the
main document.  Their contents are concatenated and the resulting file
is executed once `\begin{document}` is reached.

\DescribeEnv{gmltex} Finally, \LaTeX\ declarations that do not belong
to the preamble may be inserted in the current `gmlfile` via a
`gmltex` environment.  In the previous example, a `\large` declaration
is inserted in this way which applies to all following labels.


\section{Miscellaneous \LaTeX\ commands}
\label{latex-misc}
Apart from the environments described in the previous section, there
are only a few additional \LaTeX\ commands defined by \GML.  They
provide an interface to certain \GML\ declarations which are
frequently encountered.  Such declarations must be placed inside a
`gmlfile` environment, before the `gmlfigure` or `gmlgraph` where
their effect is needed.

From now on, we do not show the document header in the examples.  It
is understood that they are part of a \LaTeX\ document, wrapped in an
appropriate `gmlfile` environment.

\subsection{Function definitions}
\DescribeMacro{\gmlfunction} The `\gmlfunction` macro has three
arguments: the function name, a dummy variable, and a \GML\
floating-point expression which provides the function definition.  The
following example shows how to plot the function $f(x)=e^{-x/10}\cos
x$ between $x=0$ and $x=10$:
\begin{quote}
`\gmlfunction{f}{x}{exp(neg x over #10) times cos(x)}`\\
`\begin{gmlgraph}(150,80)`\\
`  fromfunction f(#0,#10): draw table plot(); endfrom`\\
`\end{gmlgraph}`\\[.5\baselineskip]
\end{quote}
\begin{center}
\gmlfunction{f}{x}{exp(neg x over #10) times cos(x)}
\begin{gmlgraph}(150,80)
  fromfunction f(#0,#10): draw table plot(); endfrom
\end{gmlgraph}
\end{center}

\subsection{Default pen scale and color}
\DescribeMacro{\gmlpenscale} \GML\ draws its lines and shapes with an
imaginary circular pen of diameter `0.5bp`.  This can be locally reset
by appropriate declarations or drawing options, but there is also a
global declaration
\begin{quote}
`\gmlpenscale{`\meta{pen diameter}`}`
\end{quote}
which is in effect for all following `gmlfigure`s and `gmlgraph`s
inside the current `gmlfile`.  The pen diameter is usually a number
(in `bp`, if not specified otherwise), but it may be any \MP\
expression which evaluates to a number.

\DescribeMacro{\gmlpencolor} Similarly, the default drawing color may
be specified by `\gmlpencolor`, which has four possible forms:
\begin{quote}
`\gmlpencolor{`\meta{R-value}`,`\meta{G-value}`,`\meta{B-value}`}`\\
`\gmlpencolor{`\meta{color expression}`}`\\
`\gmlpencolor{"rrr ggg bbb"}`\\
`\gmlpencolor{"RRGGBB"}`
\end{quote}
where the \meta{R-value} etc.\ are numbers between 0 and 1 which make
up a RGB tripel.  In the second form, \meta{color expression} is
a \MP\ color expression, allowing for named colors like
\begin{quote}
`\gmlpencolor{blue}`
\end{quote}
The string `"rrr ggg bbb"` consists of three-digit RGB values between
0 and 255, separated by single blanks.  Finally, `"RRGGBB"` stands for
a string of six hexadecimal digits.  Thus, the color
\textcolor{chartreuse}{`chartreuse`} may be defined as the current
drawing color by one of these four equivalent commands:
\begin{quote}
`\gmlpencolor{0.5, 1, 0}`\\
`\gmlpencolor{0.5red + green}`\\
`\gmlpencolor{"127 255 000}`\\
`\gmlpencolor{"7FFF00"}`
\end{quote}


\subsection{Defining new colors}
Internally, \GML\ deals with colors by the usual RGB model: each color
is represented by a tripel of numbers between 0 and 1.  The three
components represent the red, green, and blue saturation,
respectively.  Black is represented by `(0,0,0)`, white by `(1,1,1)`.
Eight basic colors are predefined as variables:
\begin{quote}
  `black`, `white`,
  \Red{`red`}, \Green{`green`}, \Blue{`blue`}, 
  \Cyan{`cyan`}, \Magenta{`magenta`}, \Yellow{`yellow`}
\end{quote}
\DescribeMacro{\gmlcolor} Additional named colors can be defined by
the `\gmlcolor` declaration.  The syntax is analogous to
`\gmlpencolor`:
\begin{quote}
`\gmlcolor{`\meta{color name}`}{`\meta{R-value}`,`\meta{G-value}%
`,`\meta{B-value}`}`\\
`\gmlcolor{`\meta{color name}`}{`\meta{color expression}`}`\\
`\gmlcolor{`\meta{color name}`}{"rrr ggg bbb"}`\\
`\gmlcolor{`\meta{color name}`}{"RRGGBB"}`
\end{quote}
After a color has been defined, it may be used as the default drawing
color 
\begin{quote}
`\gmlcolor{chartreuse}{0.5, 1, 0}`\\
`\gmlpencolor{chartreuse}`
\end{quote}
or, within a `gmlgraph` or `gmlfigure`, it can be used wherever a
color is appropriate:
\begin{quote}
`\begin{gmlfigure}`\\
`  fill fullcircle scaled 5mm withcolor chartreuse;`\\
`\end{gmlfigure}`
\end{quote}
\begin{center}
\gmlcolor{chartreuse}{0.5, 1, 0}
\begin{gmlfigure}
  fill fullcircle scaled 5mm withcolor chartreuse;
\end{gmlfigure}
\end{center}


\subsection{Color databases}
\GML\ comes with two files containing predefined colors:
`gmlcolors.tex` and `gmlextracolors.tex`.  They contain a collection
of `\gmlcolor` commands, based on the `rgb.txt` color list which is
distributed as part of the X Window system.  The color definitions may
be copied into the document, or one may simply include them
\begin{quote}
`\input gmlcolors`\\
`\input gmlextracolors`
\end{quote}


\subsection{Expanded macros in \GML\ code}
\DescribeMacro{\gml} The environments discussed so far have one
particular feature: \LaTeX\ control sequences inside the \GML\ code are
not expanded when the main document is formatted.  Usually, this is as
desired, since \LaTeX\ code in embedded labels should not be
interpreted before \GML\ has called its \LaTeX\ subjob.  However, in
some cases one would like to expand \LaTeX\ control sequences
\emph{before} code is written to file.  This is achieved by using the
`\gml` command.  For example, let us introduce a string in \GML\ which
contains the current date:
\begin{quote}
`\gml{string date; date="Today's date: \today";}`
\end{quote}
Now you can write
\begin{quote}
`\begin{gmlcode}`\\
`  message date;`\\
`\end{gmlcode}`
\end{quote}
to display today's date online when the \GML\ job is executed.


\subsection{Grouping}
In \GML, all declarations are global by
default.  The graph-drawing environments `gmlfigure` and `gmlgraph`
allow to make enclosed declarations local, if the corresponding
tokens occur in their optional argument:
\begin{quote}
`\begin{gmlfigure}[a,qqw]`\\
`  `\meta{\GML\ code}\\
`\end{gmlfigure}`
\end{quote}
Here, any macros and variable names whose first token is either `a` or
`qqw` will be local to this enviroment.  

\DescribeEnv{gmlgroup} Furthermore, there is a
generic environment
\begin{quote}
`\begin{gmlgroup}`\\
`  `\meta{\LaTeX\ and \GML\ code}\\
`\begin{gmlgroup}`
\end{quote}
which generates a surrounding group for all \GML\ code generated by
enclosed `gmlfigure`, `gmlgraph`, and `gmlcode` environments \emph{and}
for all \LaTeX\ text embedded in these graphs.

For an application, consider the `\gmlpenscale` and `\gmlpencolor`
commands described above: They are automatically local, so if you
write
\begin{quote}
`\begin{gmlgroup}`\\
`\gmlpencolor{red}`\\
`\begin{gmlfigure}`\\
`  `\meta{\GML\ code}\\
`\end{gmlfigure}`\\
`\end{gmlgroup}`
\end{quote}
the current figure is drawn with a red pen.  After the closing
`\end{gmlgroup}`, the pen color is black again (or whatever had been
its previous value).


\subsection{Turning \GML\ on and off}
\DescribeMacro{\gmlon}\DescribeMacro{\gmloff} In a long document
containing many figures, it is time-consuming to run \GML\ on the
whole input file if only a few of its figures have been changed.
Therefore, there are two switches
\begin{quote}
`\gmlon` and `\gmloff`
\end{quote}
which can be put anywhere inside a `gmlfile` environment.  When
`\gmloff` is encountered, the \GML\ interpreter is told to skip all
following `gmlfigure`s and `gmlgraph`s.  The environments `gmlcode`
and `gmltex` are unaffected.  `\gmlon` turns the interpreter on again.
The generated EPS files, if present, will be included in any case.

%%%%%%%%%%%%%%%%%%%%%%%%%%%%%%%%%%%%%%%%%%%%%%%%%%%%%%%%%%%%%%%%%%%%%%%%
\chapter{Drawing figures}
%%%%%%%%%%%%%%%%%%%%%%%%%%%%%%%%%%%%%%%%%%%%%%%%%%%%%%%%%%%%%%%%%%%%%%%%
\section{Numeric values and variables}
\subsection{Numeric type}
\GML\ has a fixed-point number system inherited from \MF.  In this
scheme, any number is represented as an integer multiple of $1/65536$.
The range is limited between $-16384$ and $+16384$.  If such numbers
are to represent physical distances on paper (measured in `bp`), this
covers ordinary paper sizes with a high accuracy.  However, numerical
values can represent other quantities as well.  There is no distinct
integer type since fixed-point numbers are representable exactly.  In
particular, in places where other programming languages accept integer
values only --- such as for loop counters and array subscripts ---
fractional values are allowed in \GML.

Variables may consist of more than one token:
\begin{quote}
`numeric a, foo.bar, xx yy zz;`
\end{quote}
Tokens (\emph{suffixes}) may be separated by a whitespace and/or a
single dot.  To declare arrays, use empty square brackets:
\begin{quote}
`numeric a[], b[][], c[]dd q[];`\\
`a3=1; b0[-1]; c[3.1]dd q0;`
\end{quote}
Numeric subscripts (which indicate array elements) may be arbitrarily
mixed with other suffixes.  Subscripts may be negative or even
fractional.  For a positive subscript the square brackets are
optional.  (The explicit `numeric` declaration is unnecessary for
numerical values, but is mandatory for other types.)

\subsection{Equations and assignment}
Variables of numeric type need not be declared explicitly.
Nevertheless, an explicit declaration is possible:
\begin{quote}
`numeric a, b, x;`\\
`a=4b; x=3; b=2x;`\\
`show a,b,x;`\\
`>> 24`\\
`>> 6`\\
`>> 3`
\end{quote}
The usual `=` sign may be used to assign variables which do not have a
value yet.  However, as this example demonstrates, the `=` operator
implies equality, not assignment.  If necessary, \MP\ automatically
solves a system of linear equations among variables to determine their
actual values.  Once this is possible, an additional equation defining
the same variable would be either redundant or inconsistent:
\begin{quote}
`b=b+1; show b;`\\
`! Inconsistent equation (off by 1).`
\end{quote}
To \emph{assign} a variable, i.e., to remove its previous value (if
any), one has to either redeclare the variable, or use the `:=`
operator:
\begin{quote}
`a:=a+1; show a;`\\
`>> 25`
\end{quote}
The `:=` assignment will remove any previous value while an `=`
assignment (i.e., equation) produces an error if a value is
accidentally overwritten.  So, in most cases the use of either form is
partly a matter of taste.  However, if one deals with point locations
on paper, the implicit equation solver is a powerful tool to describe
a graph in a rather concise and abstract way.

The equation-solving mechanism requires 
variables to allow for an \emph{unknown} state.  Variables of
any type are `unknown` before their value is fully determined (and, if they
have not been declared otherwise, they are assumed `unknown` `numeric`).  The
same is true for array elements --- therefore, arrays never have a definite
`size` or `length`.  

This feature is useful in other places: For instance, the command which
determines the extension of a graph in data coordinates may be given
unknown arguments; the appropriate values will then be determined
automatically:
\begin{quote}
`graphrange (#0,#0), (#10,??);`
\end{quote}
\DescribeMacro{??} There is one variable which is \emph{always}
`unknown`.  This is the `??` macro (a.k.a.\ `whatever`) which has been
used in the example above.

\subsection{Fixed-point arithmetics}
The basic arithmetic operations as well as some elementary
transcendental functions are available (see Table~\ref{tab:arith}):
\begin{quote}
`show 3+4, 5.3*(2.1-3), 5/2, 6**3, sqrt 2, sind 45;`\\
`>> 7`\\
`>> -4.76997`\\
`>> 2.5`\\
`>> 216.00002`\\
`>> 1.41422`\\
`>> 0.7071`
\end{quote}
Clearly, the accuracy of calculations cannot be
overwhelming, given the fact that $1/65536$ is the smallest unit.
For the applications \MP\ has originally been designed for, this is
sufficient.  Unfortunately, for the purpose of data handling, it is
not.  Therefore, floating numbers have been introduced in the \GML\
package (see below).

\subsection{Units}
To describe distances on paper, a number of constants are
predefined\footnote{To be exact, they are just ordinary variables, but
it is not a good idea to change their values.  Any named variable may
serve as a distance unit.}.  
\begin{quote}
`bp  pt  in  mm  cm`
\end{quote}
So, knowing that a multiplication sign is optional if a number is
immediately followed by a variable, distances may be expressed in
standard notation:
\begin{quote}
`show 2cm, 1.4mm, 1in;`\\
`>> 56.6929`\\
`>> 3.96848`\\
`>> 72`
\end{quote}
The default unit is `1bp`, equal to $1/72$ of an inch.

\subsection{Floating-point numbers}
The range and accuracy of fixed-point numbers is limited.  However, at
least the first limitation may be circumvented by doing calculations
with logarithms instead.  This is the approach introduced in John
Hobby's `graph.mp` macro package which has been followed by \GML.

Consequently, in \GML floating numbers do not make up a distinct type, 
but they are \emph{emulated} in this way 
on top of fixed-point numerics\footnote{A 
better solution would be to introduce floating-point numerics in the C
sources of \MP\ itself.  This, however, has not (yet?) been done.}.
In \GML\ code, a floating-point number is indicated by a preceding
hashmark:
\begin{quote}
`numeric a,b,c,d;`\\
`a = #3;  b = #.0000005;  c = #6.023e23;`\\
`showfloat a,b,c;`\\
`>> "3.00000026E+00"`\\
`>> "5.00000047E-07"`\\
`>> "6.02300063E+23"`
\end{quote}
\DescribeMacro{showfloat} The output representation of
a floating-point number is a string.  They are also input as strings
(otherwise, \MP\ would not be able to parse scientific notation such
as `#6.023e23`).  Fortunately, you do not have to type quotation marks
here: A preprocessor will insert them for you.  The `#` sign is mandatory,
however.

\DescribeMacro{\#} The operator which transforms a fixed-point number
into a floating-point one is the same `#` sign:
\begin{quote}
`numeric a,b;  a = 3;  b = #a;`\\
`show a;  showfloat b;`\\
`>> 3`\\
`>> "3.00000026E+00" )`
\end{quote}
\DescribeMacro{\$} \DescribeMacro{\$\#\$} The reverse transformation
into a number or a string is done by `#$` resp.\ `$#$`.  These two are
seldom needed.  `$#$` is the operation applied by `showfloat`.  Note
that `#$` may overflow, because there are floating-point numbers which
are not representable in the fixed-point system.

Floating-point numbers are distinguishable from fixed-point ones by
context only.  Therefore, overloading or arithmetic operators is not
possible, and a completely 
distinct system of arithmetic operations had to be defined for them.
The interpreter can't help if the two systems are unintentionally
mixed.  However, this is not critical as long as three rules are
obeyed:
\begin{enumerate}
\item Every floating-point constant appearing in \GML\ code must be
preceded by a `#` sign.
\item For integer/fixed-point numbers and variables use the ordinary
operators `+`, `-`, `*`, `/`, etc.  For floating-point numbers write
full operator names instead: `plus`, `minus`, `times`, `over` etc.
\item For debugging, use `show` to display fixed-point numbers, and
`showfloat` for floating-point numbers.
\end{enumerate}
The necessity of writing long operator names makes calculational code
less readable.  However, \GML\ should not be used for complicated
calculations anyway\footnote{Because they are inefficient and
imprecise.  Nevertheless, anything is possible in principle.}.  The
complete list of operators is found in Tab.\ref{tab:arith}.

Needless to say, automatically solving linear equations is not
implemented for floating-point numbers.  If you try it, the result will
be just garbage.


%%%%%%%%%%%%%%%%%%%%%%%%%%%%%%%%%%%%%%%%%%%%%%%%%%%%%%%%%%%%%%%%%%%%%%%%
\section{Other types}
\subsection{Pair type}
Two numeric values may be grouped into a pair:
\begin{quote}
`pair a; a=(3,2*5);`
\end{quote}
Variables of type `pair` must be declared before an assignment can be made.
Pair arrays can be defined as for numerics (or any other type):
\begin{quote}
`pair b[]x[];  b3x[-.22] = 33;`
\end{quote}
Locations on paper are described by fixed-point pair values:
\begin{quote}
`draw (0,0)--(10cm,5cm);`
\end{quote}
To access the components of a pair value, say `p`, use the `xpart` and
`ypart` operators:
\begin{quote}
`pair p; p=(3,2);`\\
`show xpart p, ypart p;`\\
`>> 3`\\
`>> 2`
\end{quote}

Pair values can mathematically be treated as vectors.  There is scalar
multiplication and division (with the `*` optional in unambiguous
cases), addition and subtraction, and a dot product:
\begin{quote}
`show 3(2,1), (4,6)/2, (2,1)+(1,-1), (2,1) dotprod (1,-1);`\\
`>> (6,3)`\\
`>> (2,3)`\\
`>> (3,0)`\\
`>> 1`
\end{quote}
The square bracket notation describes a
location like \emph{one-third on the straight line from `(2,1)` to
`(1,0)`}:
\begin{quote}
`show 1/3[(2,1), (1,0)];`\\
`>> (1.66667,0.66667)`
\end{quote}
Square brackets are also used for selecting array elements.
However, since the dimensions in a multi-dimensional array are
\emph{not} separated by commas (instead of `a[3,2]`, you must write
`a[3][2]` or `a3 2`), there is no ambiguity here.

The arithmetic operations described above are defined for fixed-point
pairs only.  Floating-point pairs are useful, nevertheless; they represent
data values: 
\begin{quote}
`draw plot((#0,#0), (#10,#5));`
\end{quote}
For floating-point pairs vector operations are not implemented:
calculations must be done on the $x$ and $y$ components separately.

\subsection{Path type}
An ordered set of points (i.e., pair values) defines a path.  In order
for \GML\ to know whether line segment are straight, joined smoothly,
or otherwise, a connection method has to be specified.  For instance,
one can define a path consisting of straight line segments
\begin{quote}
`path p; p = (0,0)--(10,5)--(4,6);`
\end{quote}
or B\'ezier curves (cubic splines)
\begin{quote}
`path p; p = (0,0)..(10,5)..(4,6);`
\end{quote}
The connection is defined by the two-character tokens `--`
and `..`, respectively.  These connectors can be mixed:
\begin{quote}
`path p; p = (0,0)..(10,5)--(4,6);`
\end{quote}
Furthermore, we should mention how to declare cyclic paths
\begin{quote}
`path q; q = (0,0)--(10,5)--(4,6)--cycle;`
\end{quote}
and how to extract the path length and to select a point within a path
\begin{quote}
`show length p,  point 1 of p,  point 1.5 of p;`\\
`>> 2`\\
`>> (10,5)`\\
`>> (7,5.50002)`
\end{quote}
Point indices count beginning from zero, and they need not be
integer.  For cyclic paths, they are cyclic, so negative values count
from the end:
\begin{quote}
`show length q,  point -1 of q;`\\
`>> 3`\\
`>> (4,6)`
\end{quote}
A subpath is selected as follows:
\begin{quote}
`tracingonline:=1; show subpath (0, 1.5) of p;`\\
`>> Path at line 0:`\\
`(0,0)..controls (3.33333,1.66667) and (6.66667,3.33333)`\\
` ..(10,5)..controls (9,5.16667) and (8,5.33334)`\\
` ..(7,5.50002)`
\end{quote}
Setting `tracingonline:=1` allows paths to be displayed on screen
(otherwise, the `show` command would write them to the logfile only).
Here, the points are listed together with (invisible) B\'ezier
control points which internally define the path shape.  There are many
ways to access these explicitly, and to get finer control on the path
shape; see the \MP\ and \MF\ manuals.

\subsection{Predefined paths}
The collection of predefined paths is helpful for designing figures and
symbols.  There are general macros for polygons, star-shaped
paths, and crosses:
\begin{quote}
`polygon` \meta{n},  `polygram` \meta{n}, and `polycross` \meta{n}
\end{quote}
where \meta{n} is an integer which specifies the number of edges.  A
couple of special cases are named
\begin{quote}
`triagon` (= `triangle`), `tetragon` (= `diamond`), `pentagon`, `hexagon`\\
`triagram`, `tetragram`, `pentagram`, `hexagram`\\
`triacross`, `tetracross` (= `cross`), `pentacross`, `hexacross`
\end{quote}
Here, `triagon` is equivalent to `polygon 3`, `tetragon` means
`polygon 4`, and so on.  Finally, there are the obvious ones
\begin{quote}
`circle` and `square`
\end{quote}
where a `circle` consists of eight points connected by B\'ezier curves.
These paths are shown in Table~\ref{tab:paths}.  Transformations
(e.g., `scaled`, `xscaled`, `yscaled`, `rotated`) and path operations
turn them into more general path shapes\footnote{For generic oval
shapes, consider the \texttt{superellipse} command described in the
\MP\ and \MF\ manuals.}.

\subsection{Pen type}
\DescribeMacro{penscale} By default, a virtual circular pen
of diameter `0.5bp` is used for drawing lines.  This can be changed by a
declaration
\begin{quote}
`penscale` \meta{pen-diameter}`;`
\end{quote}
\DescribeMacro{withpenscale}
to any value.  This declaration is local to the current figure.  To
affect the the current line only, append an option to the
corresponding drawing command:
\begin{quote}
`draw` \meta{object} `withpenscale` \meta{pen-diameter}`;`
\end{quote}
\DescribeMacro{withpen}
However, one can change not only the width, but also the \emph{shape}
of the virtual pen.  See this example:
\begin{quote}
`path p; p=(0,0){right}..(2cm,0.5cm)..(0,1cm)..{right}(2cm,1.5cm);`\\
`penscale 5mm;`\\
`draw p;`\\
`draw p shifted (3cm,0) withpen penrazor scaled 5mm;`\\
`draw p shifted (6cm,0) withpen penrazor rotated 45 scaled 5mm;`\\
`draw p shifted (9cm,0) withlinecap butt;`
\end{quote}
\begin{center}
\begin{gmlfigure}
path p; p=(0,0){right}..(2cm,0.5cm)..(0,1cm)..{right}(2cm,1.5cm);
penscale 5mm;
draw p;
draw p shifted (3cm,0) withpen penrazor scaled 5mm;
draw p shifted (6cm,0) withpen penrazor rotated 45 scaled 5mm;
draw p shifted (9cm,0) withlinecap butt;
\end{gmlfigure}
\end{center}
The same path is drawn in four different ways:  First, using a thick,
but otherwise ordinary, pen.  For the next two images we have used a
flat pen in two different orientations.  (Note
that the global `penscale` declaration applies only to the default pen.)

\DescribeMacro{pickup} A global change of pen shape and size is
achieved by a `pickup` command, which is in effect for the rest of the
current figure.  The following two commands are equivalent:
\begin{quote}
`penscale 5mm;`\\
`pickup pencircle scale 5mm;`
\end{quote}

\DescribeMacro{withlinecap} In the last image the virtual pen is
circular again, but the line ends are cut off by a `withlinecap`
option.  To enforce this globally, set the `linecap` parameter
explicitly: `linecap:=butt;` Alternatives are `butt`, `rounded`
(default), and `squared`.

In fact, a pen is a type of variable which can be generated from any
closed path.  Let us draw the path again, now using a triangular pen:
\begin{quote}
`pen tripen; tripen=makepen(triagon);`\\
`draw p withpen tripen scaled 5mm;`
\end{quote}
\begin{center}
\begin{gmlfigure}
pen tripen; tripen=makepen(triagon);
draw p withpen tripen scaled 5mm;
\end{gmlfigure}
\end{center}
\DescribeMacro{withlinejoin} Finally, the appeareance of sharp edges
is controlled by the `linejoin` parameter (globally) or the
`withlinejoin` option (locally):
\begin{quote}
`path p; p=(0,0)--(2cm,0.5cm)--(0,1cm);`\\
`penscale 5mm;`\\
`draw p withlinejoin mitered;`\\
`draw p shifted (4cm,0) withlinejoin rounded;`\\
`draw p shifted (8cm,0) withlinejoin beveled;`
\end{quote}
\begin{center}
\begin{gmlfigure}
path p; p=(0,0)--(2cm,0.5cm)--(0,1cm);
penscale 5mm;
draw p withlinejoin mitered;
draw p shifted (4cm,0) withlinejoin rounded;
draw p shifted (8cm,0) withlinejoin beveled;
\end{gmlfigure}
\end{center}


\subsection{Color type}
Colors are defined as RGB tripels.  Internally, they are treated
completely analogous to pair values, and they can be understood as
points in a three-dimensional space.  Operations that work on pair
values also apply to colors, including addition, scalar
multiplication, and even linear interpolation:
\begin{quote}
`show red, .5red, red+blue, .5[red,white];`\\
`>> (1,0,0)`\\
`>> (0.5,0,0)`\\
`>> (1,0,1)`\\
`>> (1,0.5,0.5)`
\end{quote}
When rendering colors, negative components are mapped to $0$, and
component values greater than $1$ are mapped back to $1$.

Color variables are easily defined:
\begin{quote}
`color c;  c=(.5,.1,.9);`
\end{quote}
\DescribeMacro{withcolor}
The three components of a color value or variable, say `p`, are
accessed by
\begin{quote}
`redpart p`, `greenpart p`, and `bluepart p`
\end{quote}

The `withcolor` drawing option allows for any color expression as argument.
Here, we take a predefined color:
\begin{quote}
`fill square scaled 5mm withcolor red;`\\
`draw square scaled 8mm withpenscale 2 withcolor blue;`
\end{quote}
\begin{center}
\begin{gmlfigure}
fill square scaled 5mm withcolor red;
draw square scaled 8mm withpenscale 2 withcolor blue;
\end{gmlfigure}
\end{center}
The predefined colors, as well as the \LaTeX\ interface to \GML\
colors, have been introduced above in Sec.~\ref{latex-misc}.


\subsection{Picture type}
\DescribeMacro{image} Multiple graph elements can be collected into
pictures, which by themselves can be stored in variables.
The `image` macro is a wrapper for defining pictures:
\begin{quote}
`path p; p = square scaled 1cm;`\\
`picture q; q = image(draw p; draw p rotated 45);`
\end{quote}
The result may be integrated into the final figure (which,
incidentally, is a picture variable by itself, called `currentpicture`) by
another `draw` command:
\begin{quote}
`draw q;`
\end{quote}
\begin{center}
\begin{gmlfigure}[a,p,q]
path p; p = square scaled 1cm;
picture q; q = image(draw p; draw p rotated 45);
draw q;
\end{gmlfigure}
\end{center}
\DescribeMacro{nullpicture} Operations that can be done with pictures
include assignment and junction.  \DescribeMacro{addto} The last line
in the previous example is equivalent to
\begin{quote}
`picture q;  q = nullpicture;`\\
`addto q also p;  addto q also p rotated 45;`
\end{quote}
Transformations such as `rotated` will be discussed below.


\subsection{Transform type}
\GML\ is well prepared to apply affine transformations to any
graphical object: pairs, paths, and pictures.  These include shifts,
reflections, rotations, rescalings, and more.
They may be concatenated:
\begin{quote}
`draw circle scaled 3mm;`\\
`draw circle scaled 5mm shifted (2cm,0);`\\
`draw square rotated 45;`\\
`picture p;`\\
`p=triangle scaled 5mm reflectedabout ((-1,0)--(1,0));`
\end{quote}
As a rule, in a drawing command transforms come before any drawing
options (such as `withcolor`, etc.) since they modify the object
\emph{before} it is being drawn.  If they are concatenated, they are
applied from left to right.

Transforms may be stored in variables:
\begin{quote}
`transform t; t=identity rotated 45 xscaled 2 yscaled 4;`\\
`draw q transformed t;`
\end{quote}
Here, the trivial transform `identity` serves as a starting point for
defining the transform variable.  Of course, another transform may act
on `t` afterwards, and one could apply additional transformations in
the drawing command.


\subsection{String type}
Strings are not very important in \GML.  Nevertheless, they are
available, and some operations can be done with them:
\begin{quote}
`string s,t,u;  s = "foo";  t = "bar"`\\
`u := s&t;`\\
`show u, length u, substring(1,4) of u;`\\
`>> "foobar"`\\
`>> 6`\\
`>> "oob"`
\end{quote}
As for paths, indices count beginning with zero.  Think of the
characters as corresponding to line segments, so substring indices
correspond to the points in between.

Strings may be used to display messages:
\begin{quote}
`message "Hello, world!"`
\end{quote}
More important, however, is their use as graph labels
\begin{quote}
`label("foobar", (5cm,2cm));`
\end{quote}
They are typeset in a particular font `defaultfont` (a string
variable), which is set to `"cmr10"`, `"cmr11"`, or `"cmr12"`,
depending on the default font of the enclosing \LaTeX\ document.

\subsection{\LaTeX\ strings}
\DescribeMacro{<<} \DescribeMacro{>>} The form of labels that can
be directly represented as strings is very limited, even if the
character set is extended by additional symbols (such as greek
letters, square root sign, etc.)  Fortunately, \GML\ has the full
power of \LaTeX\ at hand: Any text which is enclosed by 
`<<` and `>>` signs  is processed by \LaTeX\ and transformed into a `picture`
expression, before \GML\ comes to see it.  
So, \LaTeX\ labels can be assigned to a picture variable, or directly
be integrated into the figure.  See this example:
\begin{quote}
`picture p;  p = <<$E = mc^2$>>;`\\
`for x=0 step 1/12 until 1:`\\
`  draw p colored ((1-x)*white) shifted ((4x, x**2)*5mm);`\\
`endfor`
\end{quote}
\begin{center}
\begin{gmlfigure}
picture p;  p = <<$E = mc^2$>>;
for x=0 step 1/12 until 1:
  draw p colored ((1-x)*white) shifted ((4x, x**2)*5mm);
endfor
\end{gmlfigure}
\end{center}
The \LaTeX\ code is translated by a subprocess of \GML, which acts
as a preprocessor on the input file.  To pass declarations, packages,
and definitions to this subprocess, use the `gmltex` and `gmlpreamble`
environments described in the previous chapter.
\DescribeMacro{<<!} Alternatively, \LaTeX\ code which does not
produce output may be inserted into the \GML\ text, wrapped into the
brackets `<<!` and `>>`.


\subsection{Boolean type}
There are two boolean values
\begin{quote}
`false` and `true`
\end{quote}
and a distinct variable type `boolean` with the usual operators:
\begin{quote}
`boolean a,b;  a=true; b=false;`\\
`show a or b, not a and b;`\\
`>> true`\\
`>> false`
\end{quote}
The main use of these are control structures (see below).

%%%%%%%%%%%%%%%%%%%%%%%%%%%%%%%%%%%%%%%%%%%%%%%%%%%%%%%%%%%%%%%%%%%%%%%%
\section{Drawing commands and options}
%%%%%%%%%%%%%%%%%%%%%%%%%%%%%%%%%%%%%%%%%%%%%%%%%%%%%%%%%%%%%%%%%%%%%%%%
\section{Shapes and boxes}
%%%%%%%%%%%%%%%%%%%%%%%%%%%%%%%%%%%%%%%%%%%%%%%%%%%%%%%%%%%%%%%%%%%%%%%%
\section{Control structures}

%%%%%%%%%%%%%%%%%%%%%%%%%%%%%%%%%%%%%%%%%%%%%%%%%%%%%%%%%%%%%%%%%%%%%%%%
\chapter{Visualizing data}
\section{Floating-point numbers}
\section{Files}
\section{Functions}
\section{Datasets}
\section{Drawing commands for datasets}
\section{Scanning datasets}

\chapter{Graph appearance}
\section{Scale}
\section{Labels}
\section{Legend}
\section{Axes}
\section{Frame}

\end{gmlfile}
\end{document}

